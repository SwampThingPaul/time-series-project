\documentclass[]{article}


\begin{document}
\noindent 28-June-2017



\medskip
\medskip
\medskip
\noindent Dear editorial board of \emph{Biological Conservation}:

\medskip
\medskip


%"The cover letter should explain how the manuscript fits the scope of the journal, and more specifically how it advances the field, while having broad appeal. If the manuscript relates to any previous submission to an ESA journal, that must be explained as well.  Longer submissions those between 30 and 50 manuscript pages) should be accompanied by a detailed justification for the length. There is a required text box for the cover letter.  Uploading a cover letter as an attachment is optional" 

\indent \indent Please find enclosed the manuscript: ``Your time series is (probably) too short", by Easton R. White, to be submitted as a \emph{Short Communication} article to \emph{Biological Conservation} for consideration of publication. There is no financial interest to report. I certify that the submission is my own original work and is not under review at any other publication. A version of the submission is available as a preprint on Peerj at

Longterm census monitoring has become a cornerstone of modern ecological and conservation science. However, little work has examined the question, what is the minimum time series length required to adequately address a particular question of interest? Unfortunately, classic power analyses fail to address needs of questions related to time series analyses. I examine this question building on a few papers, in large part due to work in \emph{Biological Conservation}. In this manuscript, I use both simulation and empirical approaches to determine the minimum time required to address questions related to longterm population increases or decreases. I use an database of 878 population time series to determine that at least 10-15 years of continuous modeling are required to determine changes in abundance over time. Both simulation and empirical approaches noted the many considerations for designing a monitoring program.

I believe that my findings are of general interest to the readers of \emph{Biological Conservation} for its applications to both basic ecological research and management. This manuscript provides detailed examples of how to perform power analyses for time series using both simulation and empirical approaches. All of the R code and data used in this manuscript are freely available on Github. My work highlights the importance of longterm monitoring programs for a variety of species. The 878 populations in the manuscript represent a wide-array of life history traits and locations. This manuscript provides practical advice for designing longterm monitoring programs. 
\medskip

\noindent For potential reviewers, I would recommend:

\noindent 	- Michelle H. Reynolds, US Geological Survey, mreynolds@usgs.gov, 808-985-6416 \\
	- Jonathan R. Rhodes, University of Queensland, j.rhodes@uq.edu.au, 61 7 336 56838 \\
	- Paul C.D. Johnson, University of Glasgow, Paul.Johnson@glasgow.ac.uk, 01413306638 \\
	- Kurt E. Anderson, University of California, Riverside, kurt.anderson@ucr.edu \\


\medskip
\noindent I look forward to hearing from you at your earliest convenience.
\medskip
\medskip

Sincerely yours,

\medskip
\medskip
\medskip
\medskip

Easton White \\ 
Center for Population Biology, University of California, Davis, CA, USA, eawhite@ucdavis.edu, +1 480-203-7931






\end{document}


