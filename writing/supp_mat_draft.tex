\documentclass[12pt,]{article}
\usepackage{lmodern}
\usepackage{amssymb,amsmath}
\usepackage{ifxetex,ifluatex}
\usepackage{fixltx2e} % provides \textsubscript
\ifnum 0\ifxetex 1\fi\ifluatex 1\fi=0 % if pdftex
  \usepackage[T1]{fontenc}
  \usepackage[utf8]{inputenc}
\else % if luatex or xelatex
  \ifxetex
    \usepackage{mathspec}
  \else
    \usepackage{fontspec}
  \fi
  \defaultfontfeatures{Ligatures=TeX,Scale=MatchLowercase}
\fi
% use upquote if available, for straight quotes in verbatim environments
\IfFileExists{upquote.sty}{\usepackage{upquote}}{}
% use microtype if available
\IfFileExists{microtype.sty}{%
\usepackage{microtype}
\UseMicrotypeSet[protrusion]{basicmath} % disable protrusion for tt fonts
}{}
\usepackage[margin=1in]{geometry}
\usepackage{hyperref}
\hypersetup{unicode=true,
            pdfborder={0 0 0},
            breaklinks=true}
\urlstyle{same}  % don't use monospace font for urls
\usepackage{longtable,booktabs}
\usepackage{graphicx,grffile}
\makeatletter
\def\maxwidth{\ifdim\Gin@nat@width>\linewidth\linewidth\else\Gin@nat@width\fi}
\def\maxheight{\ifdim\Gin@nat@height>\textheight\textheight\else\Gin@nat@height\fi}
\makeatother
% Scale images if necessary, so that they will not overflow the page
% margins by default, and it is still possible to overwrite the defaults
% using explicit options in \includegraphics[width, height, ...]{}
\setkeys{Gin}{width=\maxwidth,height=\maxheight,keepaspectratio}
\IfFileExists{parskip.sty}{%
\usepackage{parskip}
}{% else
\setlength{\parindent}{0pt}
\setlength{\parskip}{6pt plus 2pt minus 1pt}
}
\setlength{\emergencystretch}{3em}  % prevent overfull lines
\providecommand{\tightlist}{%
  \setlength{\itemsep}{0pt}\setlength{\parskip}{0pt}}
\setcounter{secnumdepth}{5}
% Redefines (sub)paragraphs to behave more like sections
\ifx\paragraph\undefined\else
\let\oldparagraph\paragraph
\renewcommand{\paragraph}[1]{\oldparagraph{#1}\mbox{}}
\fi
\ifx\subparagraph\undefined\else
\let\oldsubparagraph\subparagraph
\renewcommand{\subparagraph}[1]{\oldsubparagraph{#1}\mbox{}}
\fi

%%% Use protect on footnotes to avoid problems with footnotes in titles
\let\rmarkdownfootnote\footnote%
\def\footnote{\protect\rmarkdownfootnote}

%%% Change title format to be more compact
\usepackage{titling}

% Create subtitle command for use in maketitle
\newcommand{\subtitle}[1]{
  \posttitle{
    \begin{center}\large#1\end{center}
    }
}

\setlength{\droptitle}{-2em}
  \title{}
  \pretitle{\vspace{\droptitle}}
  \posttitle{}
  \author{}
  \preauthor{}\postauthor{}
  \date{}
  \predate{}\postdate{}

\usepackage{float} \renewcommand{\thepage}{S\arabic{page}}
\renewcommand{\thesection}{S\arabic{section}}
\renewcommand{\thetable}{A\arabic{table}}
\renewcommand{\thefigure}{A\arabic{figure}} \usepackage[round]{natbib}
\bibpunct[; ]{(}{)}{,}{a}{}{;}

\begin{document}

\vspace{2cm}

\begin{center}
 \textbf{Supplementary Material: Minimum time required to detect population trends: the need for long-term monitoring programs}
 
Easton R. White
\vspace{3 mm}

Center for Population Biology, University of California, Davis, California 95616 USA

eawhite@ucdavis.edu

 \end{center}

\vspace{2cm}

\tableofcontents

\vspace{1cm}

Data and code for all the figures and tables can be found at
(\url{https://github.com/erwhite1/time-series-project}). All analyses
were run using R (R Core Team 2016).

\vspace{2cm}

\clearpage

\section{Detailed example of subsampling and power
calculations}\label{detailed-example-of-subsampling-and-power-calculations}

Here, I illustrate how we performed the subsampling and power
calculations for a specific population. As an example, I examine a
35-year time series of Bigeye tuna (\emph{Thunnus obesus}), one species
in the Global Population Dynamics Database (NERC Centre for Population
Biology 2010). Simple linear regression indicates a significant decrease
for this population with an estimated slope coefficient of -0.0271717. I
assume that this significant increase over 35 years is in fact the
``true trend``. In statistical jargon, the 35-year trend is an effect
that is actually present; we can reject the null hypothesis of no trend.
We can then use this as a benchmark to see if subsamples of the time
series also indicate a significant increase.

\begin{figure}
\centering
\includegraphics{supp_mat_draft_files/figure-latex/unnamed-chunk-2-1.pdf}
\caption{(a) Population size of Bigeye tuna (\emph{Thunnus obesus}) over
time. The line is the best fit line from linear regresssion. (b)
Statistical power for different subsets of the time series in panel
a.\label{fig:empirical_approach_example}}
\end{figure}

I then perform a subsampling routine to estimate the minimum time
required \(T_{min}\) (similar to Gerber, DeMaster, and Kareiva (1999)).
This is the same routine I used for the results in the main manuscript.

\begin{enumerate}
  \item We first extract all contiguous subsamples of the time series. This leads to 34 two-year subsamples, 33 two-year subsamples, and so forth until a single 35-year subsample.  
  \item For each subsample, we conduct linear regression and extract model coefficients and p-values.    \item We can call each set of subsamples, of the same length, a set. The proportion of subsamples within a set that show significant trends (significant slope coefficient $\alpha$ less than 0.05) is the statistical power. It is important to note that we only consider subsamples to be significant if they are significant in the same direction as the complete 35-year time series. In other words, we are conducting a one-tailed test.
  \item We can then plot statistical power as a function of time series length (Fig. \ref{fig:empirical_approach_example}b). As expected, we can see that power increases with the more years that are sampled.
  \item  Then, we determine an appropriate level of statistical power that we find acceptable. Traditionally, this has been at 0.8, however, this is purely historical. Statistical power of 0.8 implies if a true trend is present, or there is a real change in abundance, then we will detect this trend 0.8 proportion of the time. 
  \item We then determine the minimum time series length ($T_{min}$) required to achieve that level of statistical power. Here, $T_{min}$ is the first point in Fig. \ref{fig:empirical_approach_example}b where following points are also above 0.8. In this example, $T_{min}$ is 22. 
\end{enumerate}

Therefore, a minimum of 22 years of continuous monitoring are required
(for 0.8 statistical power at 0.05 significance level) to determine
long-term changes in abundance.

\pagebreak

\section{Additional results from the main
manuscript}\label{additional-results-from-the-main-manuscript}

\subsection{Correlates of minimum time
required}\label{correlates-of-minimum-time-required}

\begin{figure}
\centering
\includegraphics{supp_mat_draft_files/figure-latex/poisson_model-1.pdf}
\caption{Output of generalized linear model with a Poisson error
structure for predicting the minimum time required with explanatory
variables of the absolute value of the slope coefficient (or trend
strength), temporal autocorrelation, and coefficient of variation of
population size.\label{fig:poisson_model}}
\end{figure}

In the main text, I explained how the minimum time required strongly
correlated with the trend strength, temporal autocorrelation, and
coefficient of variation in population size. Here, I use a generalized
linear model framework with a Poisson error structure to determine
explanatory variables of the minimum time required. I use the same 822
populations as in the main text. In figure \ref{fig:poisson_model}, I
show a set of residual plots for the regression analysis. I then show
the coefficient estimates and levels of significance in table
\ref{table:model_summary_time_series_correlates}.

\begin{longtable}[]{@{}lrrrr@{}}
\caption{Output of generalized linear model to examine time series
characteristics as correlates of the minimum time required for
determining long-term population
trends.\label{table:model_summary_time_series_correlates}}\tabularnewline
\toprule
& Estimate & Std. Error & z value &
Pr(\textgreater{}\textbar{}z\textbar{})\tabularnewline
\midrule
\endfirsthead
\toprule
& Estimate & Std. Error & z value &
Pr(\textgreater{}\textbar{}z\textbar{})\tabularnewline
\midrule
\endhead
(Intercept) & 4.2658002 & 0.0369853 & 115.337838 &
0.0000000\tabularnewline
abs\_overall\_trend & -70.0413235 & 2.0027094 & -34.973283 &
0.0000000\tabularnewline
autocorrelation & 0.0601596 & 0.0550282 & 1.093249 &
0.2742843\tabularnewline
coefficient\_variation & -0.1928445 & 0.0387392 & -4.978022 &
0.0000006\tabularnewline
\bottomrule
\end{longtable}

\begin{longtable}[]{@{}lrrrr@{}}
\caption{Output of generalized linear model to examine life-history
trait correlates of the minimum time required for determining long-term
population
trends.\label{table:model_summary_biological_correlates}}\tabularnewline
\toprule
& Estimate & Std. Error & z value &
Pr(\textgreater{}\textbar{}z\textbar{})\tabularnewline
\midrule
\endfirsthead
\toprule
& Estimate & Std. Error & z value &
Pr(\textgreater{}\textbar{}z\textbar{})\tabularnewline
\midrule
\endhead
(Intercept) & 2.8131074 & 0.1297488 & 21.681181 &
0.0000000\tabularnewline
litter\_or\_clutch\_size\_n & 0.0473894 & 0.0058279 & 8.131439 &
0.0000000\tabularnewline
gen\_len & 0.0640424 & 0.0087382 & 7.328999 & 0.0000000\tabularnewline
adult\_body\_mass\_g & -0.0000809 & 0.0000177 & -4.563594 &
0.0000050\tabularnewline
maximum\_longevity\_y & -0.0074775 & 0.0019534 & -3.827907 &
0.0001292\tabularnewline
incubation\_d & -0.0127828 & 0.0031681 & -4.034864 &
0.0000546\tabularnewline
trophic2 & -0.2872678 & 0.1267799 & -2.265878 & 0.0234589\tabularnewline
trophic2.5 & -0.3604948 & 0.1260791 & -2.859274 &
0.0042461\tabularnewline
trophic3 & -0.2936987 & 0.1274698 & -2.304064 & 0.0212190\tabularnewline
\bottomrule
\end{longtable}

\begin{longtable}[]{@{}lrrrr@{}}
\caption{Output of generalized linear model to examine both time series
characteristics and life-history trait correlates of the minimum time
required for determining long-term population
trends.\label{table:model_summary_all_correlates}}\tabularnewline
\toprule
& Estimate & Std. Error & z value &
Pr(\textgreater{}\textbar{}z\textbar{})\tabularnewline
\midrule
\endfirsthead
\toprule
& Estimate & Std. Error & z value &
Pr(\textgreater{}\textbar{}z\textbar{})\tabularnewline
\midrule
\endhead
(Intercept) & 4.3236120 & 0.1361628 & 31.7532519 &
0.0000000\tabularnewline
abs(overall\_trend) & -70.8497548 & 2.8424505 & -24.9255898 &
0.0000000\tabularnewline
coefficient\_variation & -0.2089905 & 0.0531591 & -3.9314122 &
0.0000844\tabularnewline
autocorrelation & -0.0038698 & 0.0790400 & -0.0489597 &
0.9609515\tabularnewline
litter\_or\_clutch\_size\_n & 0.0008944 & 0.0058450 & 0.1530248 &
0.8783787\tabularnewline
gen\_len & 0.0087069 & 0.0088167 & 0.9875484 & 0.3233738\tabularnewline
adult\_body\_mass\_g & -0.0000214 & 0.0000174 & -1.2311406 &
0.2182703\tabularnewline
maximum\_longevity\_y & -0.0023131 & 0.0019647 & -1.1773197 &
0.2390679\tabularnewline
incubation\_d & -0.0004509 & 0.0031244 & -0.1443118 &
0.8852542\tabularnewline
trophic2 & 0.0302480 & 0.1273238 & 0.2375677 & 0.8122164\tabularnewline
trophic2.5 & -0.0101150 & 0.1263433 & -0.0800595 &
0.9361899\tabularnewline
trophic3 & -0.0049709 & 0.1276051 & -0.0389552 &
0.9689261\tabularnewline
\bottomrule
\end{longtable}

The model with trend strength, autocorrelation, and the coefficient of
variation in population size 75.13\% of the variation in the minimum
time required (Table \ref{table:model_summary_time_series_correlates}).

For a subset of populations, I examined life-history correlates as
potential explanatory variables for the minimum time required. Table
\ref{table:model_summary_biological_correlates} and
\ref{table:model_summary_all_correlates} summarize results for a model
with only life-history traits and an additional model with life-history
traits and time series characteristics combined, respectively. The model
with all life-history traits explained only 5.99\% of the variation in
the minimum time required (Table
\ref{table:model_summary_biological_correlates}). The model with all
life-history traits and time series characteristics explained 77.26\% of
the variation in the minimum time required (Table
\ref{table:model_summary_all_correlates}). When accounting for
characteristics of the time series, no life-history variables were
significant.

\pagebreak 

\pagebreak

\subsection{Minimum time series and biological
class}\label{minimum-time-series-and-biological-class}

Minimum time required differed for different biological classes. For
instance, more time is required for species within the Actinopterygii
class compared to other species (Fig. \ref{fig:class}a). These
differences between biological classes can be explained by differences
in population variability, with species in the Actinopterygii class
experiencing larger inter-annual variability in population size (Fig.
\ref{fig:class}c). In addition, populations in the Actinopterygii class
generally had weaker trends in abundance (i.e.~smaller slope
coefficients) compared to the Aves and Mammalia populations.

\begin{figure}
\centering
\includegraphics{supp_mat_draft_files/figure-latex/unnamed-chunk-8-1.pdf}
\caption{(a) Minimum time required to estimate change in abundance by
biological class, (b) long-term trend (estimated slope coefficient) by
class, (c) coefficient of variation in population size by class, and (d)
temporal autocorrelation by class.\label{fig:class}}
\end{figure}

\pagebreak

\subsection{Sensitivity analysis of significance level and
power}\label{sensitivity-analysis-of-significance-level-and-power}

Estimates of \(T_{min}\) depend strongly on the values used for the
statistical significance level (\(\alpha\)) and the probability of type
II error (\(\beta\)), both of which are set by the practitioner. Again,
statistical power is \(1-\beta\). Here we used simulations of the model
described in the main manuscript. The model simulates linear trends in
population abundance. We explored how estimates of \(T_{min}\) are
affected by changes in each of these parameters. We see that the minimum
time required increases with increases in statistical power or decreases
with increases in the threshold for statistical significance (Fig
\ref{fig:min_time_vs_alpha_beta}).

\begin{figure}
\centering
\includegraphics{supp_mat_draft_files/figure-latex/unnamed-chunk-9-1.pdf}
\caption{Minimum time required to assess long-term trends in abundance
for values of statistical significance (\(\alpha\)) and power
(\(1-\beta\)).\label{fig:min_time_vs_alpha_beta}}
\end{figure}

\subsection{Additional results for IUCN
analysis}\label{additional-results-for-iucn-analysis}

In the main text, I presented results detailing the difference between
the minimum time required to achieve a high level of statistical power
and the number of years suggested by IUCN based on generation length to
assess a species as vulnerable. There were deviations between the two
approaches. These deviations were mostly explained by the differences in
inter-annual population variability (coefficient of variation) and
generation time (Fig. \ref{fig:IUCN_correlates}).

\begin{figure}
\centering
\includegraphics{supp_mat_draft_files/figure-latex/unnamed-chunk-10-1.pdf}
\caption{The difference between minimum time estimates is the minimum
time required to achieve 0.8 statistical power versus the minimum time
required under IUCN criteria A2 to classify a species as vulnerable.
Each point represents a single population, all of which saw declines of
30\% or greater over a 10 year period. (a) Difference between minimum
time estimates versus the coefficient of variation in population size.
(b) Difference between minimum time estimates versus the generation
length in years.\label{fig:IUCN_correlates}}
\end{figure}

\pagebreak

\section{Minimum time calculations testing exponential
growth}\label{minimum-time-calculations-testing-exponential-growth}

In the main text, I evaluated the minimum time required to determine
long-term trends in abundance via linear regression. This process
examined linear trends in abundance over time. Here I examine the
minimum time required to estimate long-term trends that are either
exponential growth or decay. I use the same methods as described in the
previous sections, but we take the \(log\) of population density, or
abundance.

\begin{figure}
\centering
\includegraphics{supp_mat_draft_files/figure-latex/unnamed-chunk-11-1.pdf}
\caption{Distribution of the minimum time required in order to detect a
significant trend (at the 0.05 level) in log(abundance) given power of
0.8.\label{fig:min_time_dist_log_pop}}
\end{figure}

We see that the distribution of \(T_{min}\) is almost identical to that
in the main manuscript (Fig. \ref{fig:min_time_dist_log_pop}). This is
perhaps not surprising as most time series that would significantly
increase or decrease linearly would probably also significantly increase
or decrease at an exponential rate. Further, the calculations here and
in the main manuscript both use linear regression. Therefore both
calculations estimate the same number of parameters.

\pagebreak

\section{Simulations with more complicated population
model}\label{simulations-with-more-complicated-population-model}

In the main text, I showed how a simple population model could be
simulated repeatedly to estimate the power obtained with time series of
increasing length (Bolker 2008; Johnson et al. 2015). The model in the
main text simulated linear population growth with only a slope
coefficient, y-intercept, and noise parameter required. This model is
purely phenomenological and does not include any species life-history
information. Here, I use the same routine as the main text, but simulate
from a more biologically-realistic population model. I use the model
described in White, Nagy, and Gruber (2014). The model is a stochastic,
age-structured population model that includes density-dependence for
lemon sharks (\emph{Negaprion brevirostris}) in Bimini, Bahamas.

\begin{figure}
\centering
\includegraphics{supp_mat_draft_files/figure-latex/unnamed-chunk-12-1.pdf}
\caption{Statistical power for different length of time series
simulations for a lemon shark population in Bimini,
Bahamas.\label{fig:shark_example}}
\end{figure}

Here, I parameterize the model for a situation where adult mortality
rate is high enough to cause a population decline. As in the main text,
I simulated this more biologically-realistic model for different lengths
of time. For each length of time, I calculated the statistical power.
Similar to results with the simpler model, statistical power generally
increases with longer sampling time (Fig. \ref{fig:shark_example}). In
this example, the minimum time required (\(T_{min}\)) to obtain at least
0.8 statistical power, given a significance level of 0.05, is 27 years.

\pagebreak

\section{Using Generalized additive model to identify significant
trends}\label{using-generalized-additive-model-to-identify-significant-trends}

In the main text, I examined the minimum time required to identify a
trend in abundance via linear regression. This approach allowed us to
identify increases or decreases, but a linear model may not always be a
good fit. Generalized additive models (GAMs) are more general than
general linear models and allow more flexibility (Wood 2006). GAMs are
models where a response variable depends on unknown smooth functions of
explanatory variables. GAMs, therefore, can identify relationships
between response variables and explanatory variables that are non-linear
and perhaps more complicated. The downside of GAMs is they typically
require more data and are also prone to overfitting.

Here, I conduct the same analyses in the main text, but instead
calculate the minimum time series required to detect trends over time
according to a GAM model. I hypothesized that GAMs should require less
time to detect a trend as they are more flexible than linear regression.
I provide an example that shows statistical power increases with more
time sampled (Fig. \ref{fig:gam_example}).

\begin{figure}
\centering
\includegraphics{supp_mat_draft_files/figure-latex/unnamed-chunk-15-1.pdf}
\caption{(a) Time series for Bigeye tuna (\emph{Thunnus obesus}) with
corresponding fitted GAM model in red (with a smoothing parameter of 3)
and (b) statistical power as a function of the number of years sampled.
The horizontal ine at 0.8 indicates the minimum threshold for
statistical power and the vertical line denotes the minimum time
required to achieve 0.8 statistical power.\label{fig:gam_example}}
\end{figure}

\begin{figure}
\centering
\includegraphics{supp_mat_draft_files/figure-latex/unnamed-chunk-16-1.pdf}
\caption{Distribution of the minimum time required in order to detect a
significant trend (at the 0.05 level) in abundance according to a GAM
model given statistical power of 0.8. The smoothing parameter was set to
3 for each population.\label{fig:min_time_dist_gam}}
\end{figure}

We then fit GAM models for 799 populations. We found a similar
distribution of minimum time required as in the main text for linear
regression (Fig. \ref{fig:min_time_dist_gam}). However, in line with our
hypothesis, the GAM models did result in a lower mean minimum time
required of 14.12 years compared to the results from the main text of
15.91 years.

\clearpage

\section*{References}\label{references}
\addcontentsline{toc}{section}{References}

\hypertarget{refs}{}
\hypertarget{ref-Bolker2008}{}
Bolker, Benjamin M. 2008. \emph{Ecological Models and Data in R}. 1st
ed. Princeton, New Jersey: Princeton University Press.

\hypertarget{ref-Gerber1999}{}
Gerber, L R, D P DeMaster, and P M Kareiva. 1999. ``Gray whales and the
value of monitering data in implementing the U.S. endangered species
act.'' \emph{Conservation Biology} 13 (5): 1215--9.

\hypertarget{ref-Johnson2015}{}
Johnson, Paul CD, Sarah JE Barry, Heather M Ferguson, and Pie Müller.
2015. ``Power analysis for generalized linear mixed models in ecology
and evolution.'' \emph{Methods in Ecology and Evolution} 6 (2): 133--42.
doi:\href{https://doi.org/10.1111/2041-210X.12306}{10.1111/2041-210X.12306}.

\hypertarget{ref-GPDD2010}{}
NERC Centre for Population Biology, Imperial College. 2010. ``The Global
Population Dynamics Database Version 2.''

\hypertarget{ref-RCoreTeam2016}{}
R Core Team. 2016. ``R: A language and environment for statistical
computing.'' Vienna, Austria: R Foundation for Statistical Computing.
\url{https://www.r-project.org/}.

\hypertarget{ref-White2014}{}
White, Easton R, John D Nagy, and Samuel H Gruber. 2014. ``Modeling the
population dynamics of lemon sharks.'' \emph{Biology Direct} 9 (23):
1--18.

\hypertarget{ref-Wood2006}{}
Wood, S.N. 2006. \emph{Generalized Additive Models: An Introduction with
R}. Chapman Hall/CRC.


\end{document}
