\documentclass[12pt,]{article}
\usepackage{lmodern}
\usepackage{amssymb,amsmath}
\usepackage{ifxetex,ifluatex}
\usepackage{fixltx2e} % provides \textsubscript
\ifnum 0\ifxetex 1\fi\ifluatex 1\fi=0 % if pdftex
  \usepackage[T1]{fontenc}
  \usepackage[utf8]{inputenc}
\else % if luatex or xelatex
  \ifxetex
    \usepackage{mathspec}
  \else
    \usepackage{fontspec}
  \fi
  \defaultfontfeatures{Ligatures=TeX,Scale=MatchLowercase}
\fi
% use upquote if available, for straight quotes in verbatim environments
\IfFileExists{upquote.sty}{\usepackage{upquote}}{}
% use microtype if available
\IfFileExists{microtype.sty}{%
\usepackage{microtype}
\UseMicrotypeSet[protrusion]{basicmath} % disable protrusion for tt fonts
}{}
\usepackage[margin=1in]{geometry}
\usepackage{hyperref}
\hypersetup{unicode=true,
            pdfborder={0 0 0},
            breaklinks=true}
\urlstyle{same}  % don't use monospace font for urls
\usepackage{longtable,booktabs}
\usepackage{graphicx,grffile}
\makeatletter
\def\maxwidth{\ifdim\Gin@nat@width>\linewidth\linewidth\else\Gin@nat@width\fi}
\def\maxheight{\ifdim\Gin@nat@height>\textheight\textheight\else\Gin@nat@height\fi}
\makeatother
% Scale images if necessary, so that they will not overflow the page
% margins by default, and it is still possible to overwrite the defaults
% using explicit options in \includegraphics[width, height, ...]{}
\setkeys{Gin}{width=\maxwidth,height=\maxheight,keepaspectratio}
\IfFileExists{parskip.sty}{%
\usepackage{parskip}
}{% else
\setlength{\parindent}{0pt}
\setlength{\parskip}{6pt plus 2pt minus 1pt}
}
\setlength{\emergencystretch}{3em}  % prevent overfull lines
\providecommand{\tightlist}{%
  \setlength{\itemsep}{0pt}\setlength{\parskip}{0pt}}
\setcounter{secnumdepth}{5}
% Redefines (sub)paragraphs to behave more like sections
\ifx\paragraph\undefined\else
\let\oldparagraph\paragraph
\renewcommand{\paragraph}[1]{\oldparagraph{#1}\mbox{}}
\fi
\ifx\subparagraph\undefined\else
\let\oldsubparagraph\subparagraph
\renewcommand{\subparagraph}[1]{\oldsubparagraph{#1}\mbox{}}
\fi

%%% Use protect on footnotes to avoid problems with footnotes in titles
\let\rmarkdownfootnote\footnote%
\def\footnote{\protect\rmarkdownfootnote}

%%% Change title format to be more compact
\usepackage{titling}

% Create subtitle command for use in maketitle
\newcommand{\subtitle}[1]{
  \posttitle{
    \begin{center}\large#1\end{center}
    }
}

\setlength{\droptitle}{-2em}
  \title{}
  \pretitle{\vspace{\droptitle}}
  \posttitle{}
  \author{}
  \preauthor{}\postauthor{}
  \date{}
  \predate{}\postdate{}

\usepackage{float} \renewcommand{\thepage}{S\arabic{page}}
\renewcommand{\thesection}{S\arabic{section}}
\renewcommand{\thetable}{S\arabic{table}}
\renewcommand{\thefigure}{S\arabic{figure}} \usepackage[round]{natbib}
\bibpunct[; ]{(}{)}{,}{a}{}{;}

\begin{document}

\vspace{2cm}

\begin{center}
 \textbf{Supplementary Material: Article title here}
 
Easton R. White$^{*}$
\vspace{3 mm}

Address: \\ \emph{Center for Population Biology, University of California-Davis, Davis, California 95616 USA}

*Corresponding author: eawhite@ucdavis.edu

June 22, 2017
 \end{center}

\vspace{2cm}

\tableofcontents

\vspace{1cm}

Code for all the figures and tables can be found at
(\url{https://github.com/erwhite1}).

\vspace{2cm}

\clearpage

\section{Detailed example of subsampling and power
calculations}\label{detailed-example-of-subsampling-and-power-calculations}

Here, we illustrate how we performed the subsampling and power
calculations for a specific population. We provide more detail here than
in the main manuscript. As an example, we examine a 35-year time series
of Bigeye tuna (\emph{Thunnus obesus}), one species in the Global
Population Dynamics Database (NERC Centre for Population Biology 2010).
We assume that this significant increase over 35 years is in fact the
``true trend``. In statistical jargon, the 35-year trend is an effect
that is actually present, we can reject the null hypothesis of no trend.
We can then use this as a benchmark to see if subsamples of the time
series also indicate a significant increase.

We first extract all contiguous subsamples of the time series. This
leads to 34 two-year subsamples, 33 two-year subsamples, and so forth
until a single 35-year subsample. We can call each set of subsamples, of
the same length, a set. For each subsample, we conduct linear regression
and extract model coefficients and p-values. Then, the fraction of
subsamples within a set that show significant trends (significant slope
coefficient) is the statistical power. It is important to note that we
only consider subsamples to be significant if they are significant in
the same direction as the complete 35-year time series. In other words,
we are conducting a one-tailed test.

We can then plot statistical power as a function of time series length
(Fig. \ref{fig:empirical_approach_example}b). As expected, we can see
that power increases with the more years that are sampled. Then, we
determine an appropriate level of statistical power that we find
acceptable. Traditionally, this has been at 0.8, however, this is purely
historical. Statistical power of 0.8 implies if a true trend is present,
or a real change in abundance, then we will detect this trend 0.8
fraction of the time.

\begin{figure}[htbp]
\centering
\includegraphics{supp_mat_draft_files/figure-latex/unnamed-chunk-2-1.pdf}
\caption{(a) Population size of Bigeye tuna (\emph{Thunnus obesus}) over
time. The line is the best fit line from linear regresssion. (b)
Statistical power for different subsets of the time series in panel
a.\label{fig:empirical_approach_example}}
\end{figure}

With statistical power of 0.8, we then determine the minimum time series
length (\(T_{min}\)) required to achieve that level of statistical
power. Here, \(T_{min}\) is the first point in Fig.
\ref{fig:empirical_approach_example}b where following points are also
above 0.8. In this example, \(T_{min}\) is 22. Therefore, a minimum of
22 years of continuous monitoring are required to have 0.8 statistical
power in determining long-term changes in abundance.

\section{Additional results from the main
manuscript}\label{additional-results-from-the-main-manuscript}

\subsection{Statistical model of minimum time required and time series
characteristics}\label{statistical-model-of-minimum-time-required-and-time-series-characteristics}

In the main text, we explained how the minimum time required strongly
correlated with the trend strength, temporal autocorrelation, and
variance in population size. Here, we use a generalized linear model
framework with a Poisson error structure to determine drivers of the
minimum time required. In figure, \ref{fig:poisson_model} we show a set
of residual plots for the regression. We then show the coefficient
estimates and levels of significance in table \ref{table:model_output}.

\begin{figure}[htbp]
\centering
\includegraphics{supp_mat_draft_files/figure-latex/poisson_model-1.pdf}
\caption{Output of generalized linear model with a Poisson error
structure for predicting the minimum time required with explanatory
variables of the absolute value of the slope coefficient (or trend
strength), temporal autocorrelation, and variability in population
size.\label{fig:poisson_model}}
\end{figure}

\begin{longtable}[]{@{}lrrrr@{}}
\caption{Output of generalized linear model to examine predictors of the
minimum time required for determing long-term population
trends.\label{table:model_output}}\tabularnewline
\toprule
& Estimate & Std. Error & z value &
Pr(\textgreater{}\textbar{}z\textbar{})\tabularnewline
\midrule
\endfirsthead
\toprule
& Estimate & Std. Error & z value &
Pr(\textgreater{}\textbar{}z\textbar{})\tabularnewline
\midrule
\endhead
(Intercept) & 3.7405513 & 0.0242742 & 154.09595 & 0\tabularnewline
abs\_overall\_trend & -104.0533454 & 3.0941097 & -33.62949 &
0\tabularnewline
autocorrelation & -0.4899533 & 0.0431769 & -11.34757 & 0\tabularnewline
variance & 18.7366110 & 0.6404458 & 29.25557 & 0\tabularnewline
\bottomrule
\end{longtable}

Our model with trend strength, autocorrelation, variance, and generation
length accounted for 72.45\% of the variation in the minimum time
required (Table \ref{table:model_output}). However, we also found trend
strength and variance to be strongly correlated with one another.
Therefore, we ran two additional models with either trend strength or
variance, but not both together. This resulted in lower explained
deviance (analogue to \(R^2\)) of 53.03\% and 45.03\%, respectively.

\pagebreak 

\subsection{Minimum time required and biological
correlates}\label{minimum-time-required-and-biological-correlates}

In the main manuscript, we examined the minimum time required to detect
a significant trend in abundance over time using linear regression. As
detailed in the main manuscript the minimum time required was around 15,
but there was a wide distribution. Therefore, we were interested in
potential explanatory variables of the minimum time required. In the
main manuscript, we examined characteristics of the time series itself,
like variability, autocorrelation, and the trend in abundance over time.
Here, we combined our time series database with a database on life
history characteristics of amniotes from Myhrvold et al. (2015). There
was life history information available for 547 populations representing
315 different species, all of which were in the Aves class.

We then correlated minimum time required for each population with its
corresponding life history characteristics. In figure
\ref{fig:biological_correlates} we examined minimum time required versus
generation length (years), litter size (n), adult body mass (grams),
maximum longevity (years), egg mass (grams), and incubation (days). None
of these variables had much explanatory power in accounting for the
variance in the minimum time required. see table
\ref{table:model_output}\}

\begin{figure}[htbp]
\centering
\includegraphics{supp_mat_draft_files/figure-latex/unnamed-chunk-5-1.pdf}
\caption{Mimimum time required versus (a) generation length (years), (b)
litter size (n), (c) adult body mass (grams), (d) maximum longevity
(years), (e) egg mass (grams), and (f) incubation (days). The lines in
each plot represent the best fit line from linear
regression.\label{fig:biological_correlates}}
\end{figure}

\pagebreak

\subsection{Minimum time series and biological
class}\label{minimum-time-series-and-biological-class}

When we examined differences in the minimum time required there were
similar patterns with biological classes. For instance, more time is
required for species within the Actinopterygii class compared to other
species (Fig. \ref{fig:class}a). These differences between biological
classes can be explained by differences in population variability, with
species in Actinopterygii have larger variability in population size
from year-to-year.

\begin{figure}[htbp]
\centering
\includegraphics{supp_mat_draft_files/figure-latex/unnamed-chunk-6-1.pdf}
\caption{(a) Minimum time required to estimate change in abundance for
species class, (b) long-term trend (estimated slope coefficient) by
species class, (c) interannual variability in population size by species
class, and (d) temporal autocorrelation by species
class.\label{fig:class}}
\end{figure}

\pagebreak

\subsection{Sensitivity analysis of significance level and
power}\label{sensitivity-analysis-of-significance-level-and-power}

Estimates of \(T_{min}\) depend strongly on the values used for the
significance level (\(\alpha\)) and the probability of type II error
(\(\beta\)), both of which are set by the practitioner. Here we explore
how estimates of \(T_{min}\) are affected by changes in each of these
parameters. We see that the minimum time required increases with
increases in statistical power or decreases with increases in the
threshold for statistical significance (Fig
\ref{fig:min_time_vs_alpha_beta}).

\begin{figure}[htbp]
\centering
\includegraphics{supp_mat_draft_files/figure-latex/unnamed-chunk-7-1.pdf}
\caption{Minimum time required to assess long-term trends in abundance
for values of statistical significance (\(\alpha\)) and power
(\(1-\beta\)).\label{fig:min_time_vs_alpha_beta}}
\end{figure}

\section{Minimum time calculations testing exponential
growth}\label{minimum-time-calculations-testing-exponential-growth}

In the main text we evaluated the minimum time required to determine
long-term trends in abundance via linear regression. This process
examined linear trends in abundance over time. Here we examine the
minimum time required to estimate long-term trends that are either
exponential growth or decay. We use the same methods as described in the
previous sections, but we take the \(log\) of population density, or
abundance.

\begin{figure}[htbp]
\centering
\includegraphics{supp_mat_draft_files/figure-latex/unnamed-chunk-8-1.pdf}
\caption{Distribution of the minimum time required in order to detect a
significant trend (at the 0.05 level) in log(abundance) given power of
0.8.\label{fig:min_time_dist_log_pop}}
\end{figure}

We see that the distribution of \(T_{min}\) is almost identical to that
in the main manuscript. This is perhaps not surprising as most time
series that would significantly increase or decrease linearly would
probably also significantly increase or decrease at an exponential rate.
Further, the calculations here and in the main manuscript both use
linear regression, and both estimate the same number of parameters.

\pagebreak

\section{Simulations with more complicated population
model}\label{simulations-with-more-complicated-population-model}

In the main text, we showed how a simple population model could be
simulated repeatedly to estimate the power obtained with time series of
increasing length. The model in the main text simulated linear
population growth with only a slope coefficient, y-intercept, and noise
parameter required. This model is purely phenomenological and does not
include any species life history. Here, we use the same routine as the
main text, but simulate from a more biologically-realistic population
model. We simulate the model described in White, Nagy, and Gruber
(2014). The model is a stochastic, age-structured population model that
includes density-dependence for lemon sharks in Bimini, Bahamas.

\begin{figure}[htbp]
\centering
\includegraphics{supp_mat_draft_files/figure-latex/unnamed-chunk-9-1.pdf}
\caption{Statistical power for different length of time series
simulations for a lemon shark population in Bimini,
Bahamas.\label{fig:shark_example}}
\end{figure}

Here, we parameterize the model for a situation where adult mortality
rate is high enough to cause a population decline. In the same way we
simulated models in the main text, we simulate this more
biologically-realistic model for different lengths of time. For each
length of time, we calculate the statistical power. Similar to results
with the simpler model, statistical power generally increases with
longer sampling time (Fig. \ref{fig:shark_example}). In this example,
the minimum time required (\(T_{min}\)) to obtain at least 0.8
statistical power, given a significance level of 0.05, is 27 years.

\section{Minimum time required to estimate geometric growth
rate}\label{minimum-time-required-to-estimate-geometric-growth-rate}

Instead of detecting a trend over time with linear regression, we could
also calculate the geometric growth rate of the population. In figure
\ref{fig:growth_rate}, we show how to calculate growth rates for
subsamples of a time series. First, we created subsamples of each
possible length from the full 35 year time series, as we did in the main
next. Next, we calculated the mean and standard deviation of growth
rates for each possible time series length (Fig.
\ref{fig:growth_rate}b). Lastly, we calculated the percent error
\(\mbox{percent error} = 100 \times \left| \frac{\mbox{observed} - \mbox{theoretical}}{\mbox{theoretical}} \right|\)
between the mean of each time series length (observed) and the overall
population growth rate (theoretical). In Fig. \ref{fig:growth_rate}c),
we show the percent error as a function of time series length. Here we
define the minimum time required as the minimum number of years to
achieve less than 20\% error.

\begin{figure}[htbp]
\centering
\includegraphics{supp_mat_draft_files/figure-latex/unnamed-chunk-10-1.pdf}
\caption{Example of calculating minimum time required for growth rate
estimation. (a) European herring gull (\emph{Larus argentatus}) scaled
density over time, (b) mean and standard deviation of growth rate for
subsamples of entire time series, and (c) the percent error between mean
estimated growth rate and the true long-term growth rate. The vertical
bar denotes the minimum time required to estimate growth rate within
20\% error.\label{fig:growth_rate}}
\end{figure}

We applied the same calculations to 1032 populations, as we did in the
main text. We then obtain a distribution of the minimum time required to
measure the ``true" long-term growth rate (Fig.
\ref{fig:min_time_growth_dist}). We see a bimodal distribution with many
populations required 30+ years to estimate the long-term growth rate.
The large number of short years required is due to cases where the
entire time series is consistently increasing or decreasing at the same
rate.

\begin{figure}[htbp]
\centering
\includegraphics{supp_mat_draft_files/figure-latex/unnamed-chunk-11-1.pdf}
\caption{Histogram of the minimum time required in order to estimate the
long-term growth rate within 20\%
error.\label{fig:min_time_growth_dist}}
\end{figure}

\clearpage 

\section{Using Generalized additive model to identify significant
trends}\label{using-generalized-additive-model-to-identify-significant-trends}

In the main text, we examined the minimum time required to identify a
trend in abundance via linear regression. This approach allowed us to
identify increases or decreases, but a linear model may not always be a
good fit. Generalized additive models (GAMs) are more general than
general linear models and allow more flexible. GAMs are models where a
response variable depends on unknown smooth functions of explanatory
variables. GAMs, therefore, can identify relationships between response
variables and explanatory variables that are non-linear and perhaps more
complicated. The downside of GAMs is they typically require more data
and are also prone to overfitting (citations).

Here, we conduct the same analyses in the main text, but instead
calculate the minimum time series required to detect trends over time
according to a GAM model. We hypothesize that GAMs should require less
time to detect a trend as they are more flexible than linear regression.
We provide an example that shows statistical power increases with more
time sampled (Fig. \ref{fig:gam_example}).

\begin{figure}[htbp]
\centering
\includegraphics{supp_mat_draft_files/figure-latex/unnamed-chunk-12-1.pdf}
\caption{(a) Time series for Bigeye tuna (\emph{Thunnus obesus}) with
corresponding fitted GAM model in red and (b) statistical power as a
function of the number of years sampled. The horizontal ine at 0.8
indicates the minimum threshold for statistical power and the vertical
line denotes the minimum time required to achieve 0.8 statistical
power.\label{fig:gam_example}}
\end{figure}

\begin{figure}[htbp]
\centering
\includegraphics{supp_mat_draft_files/figure-latex/unnamed-chunk-13-1.pdf}
\caption{Distribution of the minimum time required in order to detect a
significant trend (at the 0.05 level) in abundance according to a GAM
model given statistical power of 0.8. The smoothing parameter was set to
3 for each population.\label{fig:min_time_dist_gam}}
\end{figure}

We then fit GAM models for 851 populations. We found a similar
distribution of minimum time required as in the main text for linear
regression (Fig. \ref{fig:min_time_dist_gam}). However, in line with our
hypothesis, the GAM models did result in a lower mean minimum time
required of 14.65 years compared to the results from the main text of
16.45 years.

\clearpage

\section*{References}\label{references}
\addcontentsline{toc}{section}{References}

\hypertarget{refs}{}
\hypertarget{ref-Myhrvold2015}{}
Myhrvold, Nathan P., Elita Baldridge, Benjamin Chan, Dhileep Sivam,
Daniel L. Freeman, and S.K. Morgan Ernest. 2015. ``An amniote
life-history database to perform comparative analyses with birds,
mammals, and reptiles.'' \emph{Ecology} 96 (11): 3109.

\hypertarget{ref-GPDD2010}{}
NERC Centre for Population Biology, Imperial College. 2010. ``The Global
Population Dynamics Database Version 2.''

\hypertarget{ref-White2014}{}
White, Easton R, John D Nagy, and Samuel H Gruber. 2014. ``Modeling the
population dynamics of lemon sharks.'' \emph{Biology Direct} 9 (23):
1--18.


\end{document}
