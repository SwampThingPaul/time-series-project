\documentclass[12pt,]{article}
\usepackage{lmodern}
\usepackage{amssymb,amsmath}
\usepackage{ifxetex,ifluatex}
\usepackage{fixltx2e} % provides \textsubscript
\ifnum 0\ifxetex 1\fi\ifluatex 1\fi=0 % if pdftex
  \usepackage[T1]{fontenc}
  \usepackage[utf8]{inputenc}
\else % if luatex or xelatex
  \ifxetex
    \usepackage{mathspec}
  \else
    \usepackage{fontspec}
  \fi
  \defaultfontfeatures{Ligatures=TeX,Scale=MatchLowercase}
\fi
% use upquote if available, for straight quotes in verbatim environments
\IfFileExists{upquote.sty}{\usepackage{upquote}}{}
% use microtype if available
\IfFileExists{microtype.sty}{%
\usepackage{microtype}
\UseMicrotypeSet[protrusion]{basicmath} % disable protrusion for tt fonts
}{}
\usepackage[margin=1in]{geometry}
\usepackage{hyperref}
\hypersetup{unicode=true,
            pdfborder={0 0 0},
            breaklinks=true}
\urlstyle{same}  % don't use monospace font for urls
\usepackage{graphicx,grffile}
\makeatletter
\def\maxwidth{\ifdim\Gin@nat@width>\linewidth\linewidth\else\Gin@nat@width\fi}
\def\maxheight{\ifdim\Gin@nat@height>\textheight\textheight\else\Gin@nat@height\fi}
\makeatother
% Scale images if necessary, so that they will not overflow the page
% margins by default, and it is still possible to overwrite the defaults
% using explicit options in \includegraphics[width, height, ...]{}
\setkeys{Gin}{width=\maxwidth,height=\maxheight,keepaspectratio}
\IfFileExists{parskip.sty}{%
\usepackage{parskip}
}{% else
\setlength{\parindent}{0pt}
\setlength{\parskip}{6pt plus 2pt minus 1pt}
}
\setlength{\emergencystretch}{3em}  % prevent overfull lines
\providecommand{\tightlist}{%
  \setlength{\itemsep}{0pt}\setlength{\parskip}{0pt}}
\setcounter{secnumdepth}{0}
% Redefines (sub)paragraphs to behave more like sections
\ifx\paragraph\undefined\else
\let\oldparagraph\paragraph
\renewcommand{\paragraph}[1]{\oldparagraph{#1}\mbox{}}
\fi
\ifx\subparagraph\undefined\else
\let\oldsubparagraph\subparagraph
\renewcommand{\subparagraph}[1]{\oldsubparagraph{#1}\mbox{}}
\fi

%%% Use protect on footnotes to avoid problems with footnotes in titles
\let\rmarkdownfootnote\footnote%
\def\footnote{\protect\rmarkdownfootnote}

%%% Change title format to be more compact
\usepackage{titling}

% Create subtitle command for use in maketitle
\newcommand{\subtitle}[1]{
  \posttitle{
    \begin{center}\large#1\end{center}
    }
}

\setlength{\droptitle}{-2em}
  \title{}
  \pretitle{\vspace{\droptitle}}
  \posttitle{}
  \author{}
  \preauthor{}\postauthor{}
  \date{}
  \predate{}\postdate{}

\usepackage{float} \usepackage{lineno}
\usepackage{setspace}\doublespacing \usepackage[round]{natbib}
\bibpunct[; ]{(}{)}{,}{a}{}{,}

\begin{document}

Title: Minimum time required to detect population trends: the need for
long-term monitoring programs \vspace{7 mm}

Author: \textsc{Easton R. White$^{1,2}$} \vspace{3 mm}

Address:\emph{
    \\$^1$Center for Population Biology \\
    University of California, Davis \\
    2320 Storer Hall \\
        University of California, Davis \\
        One Shields Avenue Davis, CA, USA} \vspace{3 mm}

\(^2\)Corresponding author:
\href{mailto:eawhite@ucdavis.edu}{\nolinkurl{eawhite@ucdavis.edu}} , 1
480 203 7931 \vspace{1 mm}

To be submitted to: \emph{BioScience} as a Forum \vspace{1 mm}

Number of words: 4,400 (21 pages)\\
\vspace{1 mm}

Number of figures: 4 \vspace{1 mm}

Number of references: 34 \vspace{1 mm}

Supplementary material: 9 figures and 3 tables \vspace{2 mm}

All data and code is freely available at:
(\url{https://github.com/erwhite1/time-series-project}). \vspace{2 mm}

Available as a preprint at: \url{https://peerj.com/preprints/3168/}
\vspace{2 mm}

\pagebreak

\linenumbers

\begin{abstract}
Long-term time series are necessary to better understand population dynamics, assess species' conservation status, and make management decisions. However, population data are often expensive, requiring a lot of time and resources. What is the minimum population time series length required to detect significant trends in abundance? I first present an overview of the theory and past work that has tried to address this question. As a test of these approaches, I then examine 822 populations of vertebrate species. I show that on average 15.9 years of continuous monitoring are required in order to achieve a high level of statistical power. However, there is a wide distribution around this average, casting doubt on commonly used, simple rules of thumb. These results point to the importance of long-term monitoring. I argue that statistical power needs to be considered more often in monitoring programs. Short time series are likely under-powered and potentially misleading.
\end{abstract}

Keywords: ecological time series, experimental design, population
monitoring, statistical power, sampling design

\section{Introduction}\label{introduction}

Observational studies and population time series have become a
cornerstone of modern ecological research and conservation biology
(Magurran \emph{et al.} 2010, Peters 2010, Hughes \emph{et al.} 2017).
Long-term data are necessary to both understand population dynamics and
to assess species extinction risk. Some time series may now be
considered ``long-term'' (e.g.~continuous plankton recorder, Giron-Nava
\emph{et al.} (2017)), but most are still short. Time series are
typically short due to short funding cycles and typical experimental
time-frames (Field \emph{et al.} 2007, Hughes \emph{et al.} 2017).

How long of a time series is actually necessary? This question has
important implications for both research and management (Nichols \&
Williams 2006). A short time series may lead to wrong conclusions given
large natural year-to-year variability (McCain \emph{et al.} 2016).
Managers need to know when action is needed for a population. Therefore,
managers must understand when a population trend over time is actually
meaningful. For example, the International Union for Conservation of
Nature (IUCN) Red List Categories and Criteria suggest, under Criterion
A2, a species qualifies as vulnerable if it has experienced a 30\%
decline over 10 years, or three generations (IUCN 2012). Sampling is
typically expensive, therefore, we also do not want to sample for longer
than is necessary. For example, Gerber \emph{et al.} (1999) investigated
the minimum time series required to estimate population growth of the
endangered, but recovering, eastern North Pacific gray whale
(\emph{Eschrichtius robustus}). They used a long-term census to
retroactively determine the minimum time series required to assess
threat status. They found that only 11 years were needed, eight years
before the delisting decision was made. This highlights the importance
of estimating the minimum time series required as an earlier decision
would have saved time and money (Gerber \emph{et al.} 1999). Further,
waiting too long to decide an action can imperil a species where
management action could have been taken earlier (Martin \emph{et al.}
2012, 2017).

Past work has investigated questions related to the minimum time series
required to estimate trends in population size over time (Wagner
\emph{et al.} 2009, Giron-Nava \emph{et al.} 2017). For example, Rhodes
\& Jonzen (2011) examined the optimal allocation of effort between
spatial and temporal replicates. Using simple populations models, they
found that the allocation of effort depends on environmental variation,
spatial and temporal autocorrelation, and observer error. Rueda-Cediel
\emph{et al.} (2015) also used a modeling approach, but parameterized a
model specific for a threatened snail, \emph{Tasmaphena lamproides}.
They found that for this short-lived organism, 15 years was adequate to
assess long-term trends in abundance. However, these studies, and other
past work, have typically been only on theoretical aspects of monitoring
design or focused on a few species.

Statistical power is not a new tool (Cohen 1992, Gibbs \emph{et al.}
1998), but it is still under-appreciated in ecological research (Legg \&
Nagy 2006). Therefore, I begin by reviewing key concepts of power
analyses in relation to time series analysis. I then explain how
simulation approaches have been used to estimate the minimum time
required to estimate long-term population trends. Lastly, I take an
empirical approach to estimate the minimum time required for 822 animal
populations.

\section{Statistical power in time series
analyses}\label{statistical-power-in-time-series-analyses}

An important step in experimental design is to determine the number of
samples required. For any particular experiment four quantities are
intricately linked: significance level (\(\alpha\)), statistical power,
effect size, and sample size (Legg \& Nagy 2006). The exact relationship
between these quantities depends on the specific statistical test. A
type I error is a false positive, or incorrect rejection of a true null
hypothesis. For example, if a time series was assessed as significantly
increasing or decreasing---when there was no true significant
trend---this would be a false positive. The false positive rate, or
significance level (\(\alpha\)) is often set at 0.05 (although this is
purely historical, Mapstone (1995)). A type II error (\(\beta\)) is a
failure to detect a true trend, or failure to reject a false null
hypothesis. Formally, statistical power (\(1-\beta\)) is one minus the
probability of a type II error (\(\beta\)). The effect size is a
estimate of the strength of a particular phenomenon.

Prior to an experiment, one could set appropriate levels of power,
significance level, and the effect size to estimate the sample size
required for the experiment. This approach, however, is not
straight-forward for a time series, or more complicated scenarios
(Johnson \emph{et al.} 2015), as data are clearly non-independent.

In the context of time series data, sample size can be the number of
study sites surveyed, frequency of surveys per year, and the number of
years surveyed. For example, Gibbs \emph{et al.} (1998) examined how
many times within a year a population needs to be sampled to ensure high
statistical power. They found that the sampling intensity within a year
differed greatly depending on species, because of differences in
population variability. I explore similar methods, but I focus on the
number of years required to estimate trends in abundance.

\section{Simulation approach}\label{simulation-approach}

One approach to determining the minimum time series length needed is
through repetitive simulations of a population model (Gerrodette 1987,
Gibbs \emph{et al.} 1998). This is the same approach one might use in
sample size calculations for any experimental design too complicated for
simple power analyses (Bolker 2008, Johnson \emph{et al.} 2015).
Essentially, a population model, with a selected set of parameter
values, is simulated repetitively for a number of years. As an example,
we can take the following population model for population size \(N\) at
time \(t\):

\begin{equation}
N(t + 1) = N(t) + r(t) + \epsilon \mbox{ with } \epsilon \sim N(\mu, \sigma)
\label{sim_model}
\end{equation}

where \(\epsilon\) is a normally-distributed random noise term with mean
\(\mu\) and standard deviation \(\sigma\). The rate of growth (\(r\)) is
the trend strength of the increase or decrease (i.e.~the estimated slope
from linear regression).

With regard to detecting time series trends, statistical power is the
proportion of simulations where the slope parameter from linear
regression is significant at the 0.05 threshold. Statistical power of
0.8 would indicate that, if there was a true trend in abundance, there
would be a 0.8 probability of detecting the trend. Values of 0.05 for
the significance level and 0.8 statistical power are historical (Cohen
1992) and it is important to examine the effect of changing these values
(Fig. \ref{fig:min_time_vs_alpha_beta}).

In Fig. \ref{fig:theoretical_approach}a, the model in equation
\ref{sim_model} was simulated repetitively. Statistical power increases
with increases in the length of time sampled (Fig.
\ref{fig:theoretical_approach}b). Where power is greater than 0.8 (the
dotted line), that is the minimum time required (\(T_{min}\)) to be
confident in the detection of a long-term trend in abundance. As shown
previously (Rhodes \& Jonzen 2011, Rueda-Cediel \emph{et al.} 2015),
statistical power increases with larger trend strength and lower
population variability (Fig. \ref{fig:theoretical_approach}c,d).
Simulation approaches can be useful before designing a monitoring
program or when a realistic model exists for the population in question.
However, it is also informative to examine how statistical power changes
with sampling effort in empirical time series.

It is important to note that any population model could be used here
(see example in Fig. \ref{fig:min_time_dist_log_pop}). Ideally, the
specific model choice should be tailored to the population of interest.
As an example, I determined the minimum time required to estimate
long-term population trends using a stochastic, age-structured model of
lemon shark population dynamics in the Bahamas (White \emph{et al.}
2014). I found that over 27 years of continuous monitoring were needed
in this particular scenario (Fig. \ref{fig:shark_example}). Not
surprisingly, the minimum time required for the lemon shark population
was strongly dependent on model parameters. Similarly, Rueda-Cediel
\emph{et al.} (2015) used a matrix model parameterized for a particular
snail species. They used the model to argue that only 10-15 years were
needed to accurately assess trends in abundance.

\section{Empirical approach}\label{empirical-approach}

As an empirical test of these ideas, I used a database of 2444
population time series compiled by Keith \emph{et al.} (2015). The data
are originally from the Global Population Dynamics Database (NERC Centre
for Population Biology 2010) and several other sources (Keith \emph{et
al.} 2015). I filtered out short time series (less than 35 years), and
those with missing data, leaving 822 time series. The data includes
information on 477 vertebrate species with a focus on mammals, birds,
and fish.

I assumed that each time series was long enough to include all necessary
information (e.g.~variability) about the population. In other words,
each time series was a representative sample. I first took all possible
contiguous subsamples of each time series. For example, a time series of
35 years had 34 possible contiguous subsamples of length 2, 34 possible
contiguous subsamples of length 3, and continuing until 1 possible
contiguous subsample of length 35 (Gerber \emph{et al.} 1999, Brashares
\& Sam 2005, Giron-Nava \emph{et al.} 2017). In line with the simulation
approach, I determined the proportion of subsamples of a particular
length that had estimated slope coefficients statistically different
from zero. This proportion is a measure of statistical power. Lastly, I
determined which subsample length is required to achieve a certain
threshold of statistical power (0.8, Cohen (1992)). The minimum
subsampled length that met these criteria was the minimum time series
length required (\(T_{min}\)). All analyses were conducted in R (R Core
Team 2017).

\subsection{Estimates of the minimum time
required}\label{estimates-of-the-minimum-time-required}

Across all the populations, I found an average minimum time series
length required (\(T_{min}\)) of 15.9 (SD=8.3), with a wide distribution
(Fig. \ref{fig:min_time_dist}b). Estimates of \(T_{min}\) varied between
biological class (Fig. \ref{fig:min_time_dist}a). Ray-finned fish (class
Actinopterygii) typically had estimates of \(T_{min}\) over 20 years.
Birds (class Aves) had a much wider distribution of \(T_{min}\), but
usually required less years of sampling. Differences between these
classes can be explained by differences in variability in population
size and strength of trends in abundance (Fig. \ref{fig:class}).

This time-frame is in line with past work on a short-lived snail species
(Rueda-Cediel \emph{et al.} 2015) and a long-lived whale species (Gerber
\emph{et al.} 1999). Hatch (2003) used seabird monitoring data to
estimate minimum sampling requirements. He found that the time required
ranged from 11 to 69 years depending on species, trend strength, and
study design.

\subsection{Corrrelates for minimum time
required}\label{corrrelates-for-minimum-time-required}

The minimum time series length required was strongly correlated with
trend strength (i.e.~estimated slope coefficient from linear
regression), coefficient of variation in population size, and
autocorrelation in population size (Figs. \ref{fig:correlates}a-c). All
three of these explanatory variables were significant and had large
effect sizes (see Table A1). Combined, trend strength, coefficient of
variation in population, and autocorrelation account for 75.1\% of the
explained deviance (Zuur \emph{et al.} 2009) in minimum time series
length required.

There was life-history information available (Myhrvold \emph{et al.}
2015) for 547 populations representing 315 different species, all of
which were birds (Aves class). Some life-history traits were significant
predictors for the minimum time required (Figs. \ref{fig:correlates}d-h,
Tables A2,A3). However, even combined, all five of the life-history
traits accounted for only 5.99\% of the explained deviance in minimum
time series length required. In addition, when accounting for trend
strength, coefficient of variation, and autocorrelation, no life-history
traits were significant predictors of the minimum time required (Table
A3).

I initially hypothesized that species with longer lifespans or
generation times may require a longer sampling period. This result could
have been a result of at least two factors. First, the data I used may
not include a diverse enough set of species with different life-history
traits. Second, the question I posed, whether a population is increasing
or decreasing, was specifically concerned with population trends over
time. Therefore, life-history characteristics may be more important for
other questions more closely tied to species biology. For example,
Blanchard \emph{et al.} (2007) used detailed simulations of
spatially-distributed fisheries to compare survey designs. They found
that statistical power depended on survey design, temperature
preferences, and the degree of population patchiness.

\subsection{Evalulating the IUCN
criteria}\label{evalulating-the-iucn-criteria}

I examined a subset of populations with observed declines of 30\% or
greater over 10 years, qualifying all of them as vulnerable under IUCN
Criterion A2 (IUCN 2012). This resulted in n = 162 populations. I then
compared the minimum time required to achieve 0.8 statistical power
(\(T_{min}\)) to the minimum time required under the IUCN criteria (Fig.
\ref{fig:IUCN_analysis}). For populations below the identity line in
figure \ref{fig:IUCN_analysis}, IUCN criteria would require more
sampling compared to estimates for \(T_{min}\). Further, populations
above the identity line are cases where the IUCN criteria would classify
a population as vulnerable despite not having sampled enough years to
achieve high statistical power (Fig. \ref{fig:IUCN_analysis}). The
silhouettes on figure \ref{fig:IUCN_analysis} highlight that species
with long generation times had larger discrepancies between \(T_{min}\)
and the minimum time required for IUCN assessments (Fig.
\ref{fig:IUCN_correlates}).

For the populations I examined, the IUCN criteria may be overly
simplistic (Fig. \ref{fig:IUCN_analysis}). For many populations, the
IUCN criteria suggest more years than necessary are required to assess a
population as vulnerable. Conversely, for other populations the IUCN
criteria suggest sampling times that are less than the minimum time
required for statistical power. This suggests that the IUCN criteria are
probably too simplistic as the minimum time required does not correlate
with generation time or other biological covariates (Fig.
\ref{fig:correlates}d-h).

\section{Related questions}\label{related-questions}

An important related idea is the optimal allocation of sampling effort
in space versus time. In a theoretical investigation of this question,
Rhodes \& Jonzen (2011) found that the optimal allocation of sampling
depended strongly on temporal and spatial autocorrelation. If spatial
population dynamics were highly correlated, then it was better to sample
more temporally, and vice versa. The empirical data supports this idea
as populations with strong temporal autocorrelation needed less years of
sampling (Fig. \ref{fig:correlates}). Morrison \& Hik (2008) also
studied the optimal allocation of sampling effort in space versus time,
but used emprical data from a long-term survey of the collared pika
(\emph{Ochotona collaris}) in the Yukon. They found that surveys less
than 5 years may be misleading and that extrapolating from one
population to another, even when nearby geographically, may be
untenable.

Seavy \& Reynolds (2007) asked whether statistical power was even a
useful framework for assessing long-term population trends. They used 24
years of census data on Red-tailed Tropicbirds
(\emph{Phaethon rubricauda}) in Hawaii and showed that to detect a 50\%
decline over 10 years almost always resulted in high statistical power
(above 0.8). Therefore, they cautioned against only using power analyses
to design monitoring schemes and instead argued for metrics that would
increase precision: improving randomization, reducing bias, and
increasing detection probability. Power analyses should not be the only
consideration when designing monitoring schemes. However, unlike Seavy
\& Reynolds (2007), the results here indicate that longer than 10 years
is often needed to achieve high statistical power.

\section{Limitations}\label{limitations}

This paper has some limitations in determining the minimum time series
length required. First, the minimum time estimated is particular to the
specific question of interest. An additional complication is that for
the empirical approach, the subsampling of the full time series allows
for estimates of power, but the individual subsamples are clearly not
independent of one another. In an ideal setting, a specific population
model would be parameterized for each population of interest. Then,
model simulations could be used to estimate the minimum time series
required to address each specific question of interest. Clearly, this is
not always practical, especially if conducting analyses for a wide array
of species.

\section{Conclusions}\label{conclusions}

Power analyses are not a novel aspect of ecological research (Legg \&
Nagy 2006). However, power analyses are still underutilized, especially
in the context of time series analyses. I used a database of 822
populations to determine the minimum time series length required to
detect population trends. This goes beyond previous work that either
focused on theoretical investigations or a limited number of species. I
show that to identify long-term changes in abundance, on average 15.91
years of continuous monitoring are often required (Fig.
\ref{fig:min_time_dist}). However, there is wide distribution of
estimated minimum times. Therefore, it is probably not wise to use a
simple threshold number of years in monitoring design. Further, contrary
to my initial hypotheses, minimum time required did not correlate with
generation time or any other life-history traits (Fig.
\ref{fig:correlates}d-h). This result argues against overly simplified
measures of minimum sampling time based on generation length or other
life-history traits, like those of the IUCN criteria (Fig.
\ref{fig:IUCN_analysis}).

The design of monitoring programs should include calculations of
statistical power, the allocation of sampling in space versus time
(Larsen \emph{et al.} 2001, Rhodes \& Jonzen 2011), and metrics to
increase precision (Seavy \& Reynolds 2007). Ideally, a formal decision
analysis to evaluate these different factors would be conducted to
design or assess any monitoring program (Hauser \emph{et al.} 2006,
McDonald-Madden \emph{et al.} 2010). This type of formal decision
analysis would also include information on the costs of monitoring.
These costs include the actual costs of sampling (Brashares \& Sam 2005)
and the ecological costs of inaction (Thompson \emph{et al.} 2000).

For many populations, short time series are probably not reliable for
detecting population trends. This result highlights the importance of
long-term monitoring programs. From both a scientific and management
perspective, estimates of \(T_{min}\) are important. If a time series is
too short, we lack the statistical power to reliably detect long-term
population trends. In addition, a time series that is too long may be a
poor use of already limited funds (Gerber \emph{et al.} 1999). Further,
more data is not always best in situations where management actions need
to be taken (Martin \emph{et al.} 2012, 2017). When a population trend
is detected, it may be too late for management action. In these
situations, the precautionary principle may be more appropriate
(Thompson \emph{et al.} 2000). Future work should examine other species,
with a wider range of life-history characteristics. In addition, similar
approaches can be used to determine the minimum time series length
required to address additional questions of interest.

\section{Supporting Information}\label{supporting-information}

In the supporting material, I provide an expanded methods sections,
additional figures, minimum time calculations for determining
exponential growth, simulations with a more complicated population
model, and the use of generalized additive models to identify population
trends. All code and data can be found at
\url{https://github.com/erwhite1/time-series-project}

\section{Acknowledgements}\label{acknowledgements}

ERW was partially supported by a National Science Foundation Graduate
Fellowship. I would like to thank members of the Ecological Theory group
at the University of California, Davis for their insight.

\section{References}\label{references}

\hypertarget{refs}{}
\hypertarget{ref-Blanchard2007}{}
Blanchard, J.L., Maxwell, D.L. \& Jennings, S. (2007). Power of
monitoring surveys to detect abundance trends in depleted populations:
the effects of density-dependent habitat use, patchiness, and climate
change. \emph{ICES Journal of Marine Science}, 65, 111--120.

\hypertarget{ref-Bolker2008}{}
Bolker, B.M. (2008). \emph{Ecological Models and Data in R}. 1st edn.
Princeton University Press, Princeton, New Jersey.

\hypertarget{ref-Brashares2005}{}
Brashares, J.S. \& Sam, M.K. (2005). How much is enough? Estimating the
minimum sampling required for effective monitoring of African reserves.
\emph{Biodiversity and Conservation}, 14, 2709--2722.

\hypertarget{ref-Cohen1992}{}
Cohen, J. (1992). A power primer. \emph{Psychological Bulletin}, 112,
155--159.

\hypertarget{ref-Field2007}{}
Field, S.A., Connor, P.J.O., Tyre, A.J. \& Possingham, H.P. (2007).
Making monitoring meaningful. \emph{Austral Ecology}, 32, 485--491.

\hypertarget{ref-Gerber1999}{}
Gerber, L.R., DeMaster, D.P. \& Kareiva, P.M. (1999). Gray whales and
the value of monitering data in implementing the U.S. endangered species
act. \emph{Conservation Biology}, 13, 1215--1219.

\hypertarget{ref-Gerrodette1987}{}
Gerrodette, T. (1987). A power analysis for detecting trends.

\hypertarget{ref-Gibbs1998}{}
Gibbs, J.P., Droege, S. \& Eagle, P. (1998). Monitoring populations of
plants and animals. \emph{BioScience}, 48, 935--940.

\hypertarget{ref-Giron-Nava2017}{}
Giron-Nava, A., James, C.C., Johnson, A.F., Dannecker, D., Kolody, B. \&
Lee, A. \emph{et al.} (2017). Quantitative argument for long-term
ecological monitoring. \emph{Marine Ecology Progress Series}, 572,
269--274.

\hypertarget{ref-Hatch2003}{}
Hatch, S.A. (2003). Statistical power for detecting trends with
applications to seabirds monitoring. \emph{Biological Conservation},
111, 317--329.

\hypertarget{ref-Hauser2006}{}
Hauser, C.E., Pople, A.R. \& Possingham, H.P. (2006). Should managed
populations be monitored every year? \emph{Ecological Applications}, 16,
807--819.

\hypertarget{ref-Hughes2017}{}
Hughes, B.B., Beas-luna, R., Barner, A.K., Brewitt, K., Brumbaugh, D.R.
\& Cerny-Chipman, E.B. \emph{et al.} (2017). Long-term studies
contribute disproportionately to ecology and policy. \emph{BioScience},
67, 271--281.

\hypertarget{ref-IUCN2012}{}
IUCN. (2012). IUCN Red List Categories and Criteria: Version 3.1.

\hypertarget{ref-Johnson2015}{}
Johnson, P.C., Barry, S.J., Ferguson, H.M. \& Müller, P. (2015). Power
analysis for generalized linear mixed models in ecology and evolution.
\emph{Methods in Ecology and Evolution}, 6, 133--142.

\hypertarget{ref-Keith2015}{}
Keith, D., Akçakaya, H.R., Butchart, S.H., Collen, B., Dulvy, N.K. \&
Holmes, E.E. \emph{et al.} (2015). Temporal correlations in population
trends: Conservation implications from time-series analysis of diverse
animal taxa. \emph{Biological Conservation}, 192, 247--257.

\hypertarget{ref-Larsen2001}{}
Larsen, D.P., Kincaid, T.M., Jacobs, S.E. \& Urquhart, N.S. (2001).
Designs for Evaluating Local and Regional Scale Trends.
\emph{BioScience}, 51, 1069.

\hypertarget{ref-Legg2006}{}
Legg, C.J. \& Nagy, L. (2006). Why most conservation monitoring is, but
need not be, a waste of time. \emph{Journal of Environmental
Management}, 78, 194--199.

\hypertarget{ref-Magurran2010}{}
Magurran, A.E., Baillie, S.R., Buckland, S.T., Dick, J.M., Elston, D.A.
\& Scott, E.M. \emph{et al.} (2010). Long-term datasets in biodiversity
research and monitoring : assessing change in ecological communities
through time. \emph{Trends in Ecology and Evolution}, 25, 574--582.

\hypertarget{ref-Mapstone1995}{}
Mapstone, B.D. (1995). Scalable decision rules for environmental impact
studies : effect Size , type I , and type II errors. \emph{Ecological
Applications}, 5, 401--410.

\hypertarget{ref-Martin2017}{}
Martin, T.G., Camaclang, A.E., Possingham, H.P., Maguire, L.A. \&
Chadès, I. (2017). Timing of Protection of Critical Habitat Matters.
\emph{Conservation Letters}, 10, 308--316.

\hypertarget{ref-Martin2012}{}
Martin, T.G., Nally, S., Burbidge, A.A., Arnall, S., Garnett, S.T. \&
Hayward, M.W. \emph{et al.} (2012). Acting fast helps avoid extinction.
\emph{Conservation Letters}, 5, 274--280.

\hypertarget{ref-McCain2016}{}
McCain, C.M., Szewczyk, T. \& Knight, K.B. (2016). Population
variability complicates the accurate detection of climate change
responses. \emph{Global Change Biology}, 22, 2081--2093.

\hypertarget{ref-McDonald-Madden2010}{}
McDonald-Madden, E., Baxter, P.W.J., Fuller, R.A., Martin, T.G., Game,
E.T. \& Montambault, J. \emph{et al.} (2010). Monitoring does not always
count. \emph{Trends in Ecology and Evolution}, 25, 547--550.

\hypertarget{ref-Morrison2008}{}
Morrison, S. \& Hik, D.S. (2008). When? Where? And for how long? Census
design considerations for an Alpine Lagomorph, the Collared pika. In:
\emph{Lagomorph biology}. Springer Berlin Heidelberg, pp. 103--113.

\hypertarget{ref-Myhrvold2015}{}
Myhrvold, N.P., Baldridge, E., Chan, B., Sivam, D., Freeman, D.L. \&
Ernest, S.M. (2015). An amniote life-history database to perform
comparative analyses with birds, mammals, and reptiles. \emph{Ecology},
96, 3109.

\hypertarget{ref-GPDD2010}{}
NERC Centre for Population Biology, I.C. (2010). The Global Population
Dynamics Database Version 2.

\hypertarget{ref-Nichols2006}{}
Nichols, J.D. \& Williams, B.K. (2006). Monitoring for conservation.
\emph{Trends in Ecology and Evolution}, 21, 668--673.

\hypertarget{ref-Peters2010}{}
Peters, D.P. (2010). Accessible ecology: Synthesis of the long, deep,
and broad. \emph{Trends in Ecology and Evolution}, 25, 592--601.

\hypertarget{ref-RCoreTeam2017}{}
R Core Team. (2017). R: A language and environment for statistical
computing.

\hypertarget{ref-Rhodes2011}{}
Rhodes, J.R. \& Jonzen, N. (2011). Monitoring temporal trends in
spatially structured populations: how should sampling effort be
allocated between space and time? \emph{Ecography}, 34, 1040--1048.

\hypertarget{ref-Rueda-Cediel2015}{}
Rueda-Cediel, P., Anderson, K.E., Regan, T.J., Franklin, J. \& Regan, M.
(2015). Combined influences of model choice, data quality, and data
quantity when estimating population trends. \emph{PLoSONE}, 10,
e0132255.

\hypertarget{ref-Seavy2007}{}
Seavy, N.E. \& Reynolds, M.H. (2007). Is statistical power to detect
trends a good assessment of population monitoring? \emph{Biological
Conservation}, 140, 187--191.

\hypertarget{ref-Thompson2000}{}
Thompson, P.M., Wilson, B., Grellier, K. \& Hammond, P.S. (2000).
Combining power analysis and population viability analysis to compare
traditional and precautionary approaches to conservation of coastal
cetaceans. \emph{Conservation Biology}, 14, 1253--1263.

\hypertarget{ref-Wagner2009}{}
Wagner, T., Vandergoot, C.S. \& Tyson, J. (2009). Evaluating the power
to detect temporal trends in fishery-independent surveys - A case study
based on gill nets set in the Ohio waters of Lake Erie for walleyes.
\emph{North American Journal of Fisheries Management}, 29, 805--816.

\hypertarget{ref-White2014}{}
White, E.R., Nagy, J.D. \& Gruber, S.H. (2014). Modeling the population
dynamics of lemon sharks. \emph{Biology Direct}, 9, 1--18.

\hypertarget{ref-Zuur2009}{}
Zuur, A.F., Ieno, E.N., Walker, N.J., Saveliev, A.A. \& Smith, G.M.
(2009). \emph{Mixed Effects Models and Extensions in Ecology with R}.
Springer, New York.

\clearpage

\pagebreak

\section{Figure captions}\label{figure-captions}

Figure 1: (a) Example of a simulated time series for 40 time periods.
(b) Statistical power versus the simulated time series length. The
horizontal, dashed line is the desired statistical power of 0.8. The
vertical, dashed line is the minimum time required to achieve the
desired statistical power. (c) Minimum time required (\(T_{min}\)) for
simulations with different values of the trend strength (\(r\)) and
\(\sigma = 5.0\). (d) Minimum time required for different levels of
population variability (\(\sigma\)) and \(r=1.5\). In each case, the
minimum time required is the minimum number of years to achieve 0.8
statistical power given a significance level of 0.05..

Figure 2: (a) Distributions of the minimum time required for populations
from four different biological classes. (b) Distribution of minimum time
required for all populations regardless of biological class. The minimum
time required calculation corresponds to a significance level of 0.05
and statistical power of 0.8.

Figure 3: Minimum time required to estimate change in abundance
correlated with (a) trend strength (absolute value of slope coefficient
estimated from linear regression), (b) coefficient of variation in
interannual population size, (c) temporal lag-1 autocorrelation, (d)
generation length (years), (e) litter size (n), (f) log adult body mass
(grams), (g) maximum longevity (years), and (h) incubation (days). The
lines in each plot represent the best fit line from linear regression.

Figure 4: Minimum time required to achieve 0.8 statistical power versus
the minimum time required under IUCN criteria A2 to classify a species
as vulnerable. Each point represents a single population, all of which
saw declines of 30\% or greater over a 10 year period. The silhouettes
highlight that species with longer generation times typically have
larger discrepancies between \(T_{min}\) and the minimum time required
for IUCN assessments.

\pagebreak

\begin{figure}
\centering
\includegraphics{manuscript_draft_short_version_files/figure-latex/unnamed-chunk-5-1.pdf}
\caption{(a) Example of a simulated time series for 40 time periods. (b)
Statistical power versus the simulated time series length. The
horizontal, dashed line is the desired statistical power of 0.8. The
vertical, dashed line is the minimum time required to achieve the
desired statistical power. (c) Minimum time required (\(T_{min}\)) for
simulations with different values of the trend strength (\(r\)) and
\(\sigma = 5.0\). (d) Minimum time required for different levels of
population variability (\(\sigma\)) and \(r=1.5\). In each case, the
minimum time required is the minimum number of years to achieve 0.8
statistical power given a significance level of
0.05.\label{fig:theoretical_approach}}
\end{figure}

\pagebreak

\begin{figure}
\centering
\includegraphics{manuscript_draft_short_version_files/figure-latex/unnamed-chunk-6-1.pdf}
\caption{(a) Distributions of the minimum time required for populations
from four different biological classes. (b) Distribution of minimum time
required for all populations regardless of biological class. The minimum
time required calculation corresponds to a significance level of 0.05
and statistical power of 0.8.\label{fig:min_time_dist}}
\end{figure}

\clearpage

\begin{figure}
\centering
\includegraphics{manuscript_draft_short_version_files/figure-latex/unnamed-chunk-9-1.pdf}
\caption{Minimum time required to estimate change in abundance
correlated with (a) trend strength (absolute value of slope coefficient
estimated from linear regression), (b) coefficient of variation in
interannual population size, (c) temporal lag-1 autocorrelation, (d)
generation length (years), (e) litter size (n), (f) log adult body mass
(grams), (g) maximum longevity (years), and (h) incubation (days). The
lines in each plot represent the best fit line from linear
regression.\label{fig:correlates}}
\end{figure}

\clearpage

\begin{figure}
\centering
\includegraphics{manuscript_draft_short_version_files/figure-latex/unnamed-chunk-10-1.pdf}
\caption{Minimum time required to achieve 0.8 statistical power versus
the minimum time required under IUCN criteria A2 to classify a species
as vulnerable. Each point represents a single population, all of which
saw declines of 30\% or greater over a 10 year period. The silhouettes
highlight that species with longer generation times typically have
larger discrepancies between \(T_{min}\) and the minimum time required
for IUCN assessments.\label{fig:IUCN_analysis}}
\end{figure}

\clearpage

\section{Supplementary material}\label{supplementary-material}

\setcounter{figure}{0} \renewcommand{\thefigure}{A\arabic{figure}}

\begin{figure}[htbp]
\centering
\caption{(a) Population size of Bigeye tuna (\emph{Thunnus obesus}) over
time. The line is the best fit line from linear regression. (b)
Statistical power for different subsets of the time series in panel
a.\label{fig:empirical_approach_example}}
\end{figure}

\begin{figure}[htbp]
\centering
\caption{Output of generalized linear model with a Poisson error
structure for predicting the minimum time required with explanatory
variables of the absolute value of the slope coefficient (or trend
strength), temporal autocorrelation, and variability in population
size.\label{fig:poisson_model}}
\end{figure}

\begin{figure}[htbp]
\centering
\caption{(a) Minimum time required to estimate change in abundance by biological class, (b) long-term trend (estimated slope coefficient) by class, (c) coefficient of variation in population size by class, and (d) temporal autocorrelation by class.\label{fig:class}}
\end{figure}

\begin{figure}[htbp]
\centering
\caption{Minimum time required to assess long-term trends in abundance
for values of statistical significance (\(\alpha\)) and power
(\(1-\beta\)).\label{fig:min_time_vs_alpha_beta}}
\end{figure}

\begin{figure}[htbp]
\centering
\caption{The difference between minimum time estimates is the minimum time required to achieve 0.8 statistical power versus the minimum time required under IUCN criteria A2 to classify a species as vulnerable. Each point represents a single population, all of which saw declines of 30\% or greater over a 10 year period. (a) Difference between minimum time estimates versus the coefficient of variation in population size. (b) Difference between minimum time estimates versus the generation length in years.\label{fig:IUCN_correlates}}
\end{figure}

\begin{figure}[htbp]
\centering
\caption{Distribution of the minimum time required in order to detect a
significant trend (at the 0.05 level) in log(abundance) given power of
0.8.\label{fig:min_time_dist_log_pop}}
\end{figure}

\begin{figure}[htbp]
\centering
\caption{Statistical power for different length of time series
simulations for a lemon shark population in Bimini,
Bahamas.\label{fig:shark_example}}
\end{figure}

\begin{figure}[htbp]
\centering
\caption{(a) Time series for Bigeye tuna (\emph{Thunnus obesus}) with
corresponding fitted GAM model in red and (b) statistical power as a
function of the number of years sampled. The horizontal line at 0.8
indicates the minimum threshold for statistical power and the vertical
line denotes the minimum time required to achieve 0.8 statistical
power.\label{fig:gam_example}}
\end{figure}

\begin{figure}[htbp]
\centering
\caption{Distribution of the minimum time required in order to detect a
significant trend (at the 0.05 level) in abundance according to a GAM
model given statistical power of 0.8. The smoothing parameter was set to
3 for each population.\label{fig:min_time_dist_gam}}
\end{figure}

\vspace{1 cm}

Table A1: Output of generalized linear model to examine time series
characteristics as correlates of the minimum time required for
determining long-term population trends.

\vspace{1 cm}

Table A2: Output of generalized linear model to examine life-history
trait correlates of the minimum time required for determine long-term
population trends.

\vspace{1 cm}

Table A3: Output of generalized linear model to examine both time series
characteristics and life-history trait correlates of the minimum time
required for determine long-term population trends.


\end{document}
