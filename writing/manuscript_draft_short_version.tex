\documentclass[12pt,]{article}
\usepackage{lmodern}
\usepackage{amssymb,amsmath}
\usepackage{ifxetex,ifluatex}
\usepackage{fixltx2e} % provides \textsubscript
\ifnum 0\ifxetex 1\fi\ifluatex 1\fi=0 % if pdftex
  \usepackage[T1]{fontenc}
  \usepackage[utf8]{inputenc}
\else % if luatex or xelatex
  \ifxetex
    \usepackage{mathspec}
  \else
    \usepackage{fontspec}
  \fi
  \defaultfontfeatures{Ligatures=TeX,Scale=MatchLowercase}
\fi
% use upquote if available, for straight quotes in verbatim environments
\IfFileExists{upquote.sty}{\usepackage{upquote}}{}
% use microtype if available
\IfFileExists{microtype.sty}{%
\usepackage{microtype}
\UseMicrotypeSet[protrusion]{basicmath} % disable protrusion for tt fonts
}{}
\usepackage[margin=1in]{geometry}
\usepackage{hyperref}
\hypersetup{unicode=true,
            pdfborder={0 0 0},
            breaklinks=true}
\urlstyle{same}  % don't use monospace font for urls
\usepackage{graphicx,grffile}
\makeatletter
\def\maxwidth{\ifdim\Gin@nat@width>\linewidth\linewidth\else\Gin@nat@width\fi}
\def\maxheight{\ifdim\Gin@nat@height>\textheight\textheight\else\Gin@nat@height\fi}
\makeatother
% Scale images if necessary, so that they will not overflow the page
% margins by default, and it is still possible to overwrite the defaults
% using explicit options in \includegraphics[width, height, ...]{}
\setkeys{Gin}{width=\maxwidth,height=\maxheight,keepaspectratio}
\IfFileExists{parskip.sty}{%
\usepackage{parskip}
}{% else
\setlength{\parindent}{0pt}
\setlength{\parskip}{6pt plus 2pt minus 1pt}
}
\setlength{\emergencystretch}{3em}  % prevent overfull lines
\providecommand{\tightlist}{%
  \setlength{\itemsep}{0pt}\setlength{\parskip}{0pt}}
\setcounter{secnumdepth}{5}
% Redefines (sub)paragraphs to behave more like sections
\ifx\paragraph\undefined\else
\let\oldparagraph\paragraph
\renewcommand{\paragraph}[1]{\oldparagraph{#1}\mbox{}}
\fi
\ifx\subparagraph\undefined\else
\let\oldsubparagraph\subparagraph
\renewcommand{\subparagraph}[1]{\oldsubparagraph{#1}\mbox{}}
\fi

%%% Use protect on footnotes to avoid problems with footnotes in titles
\let\rmarkdownfootnote\footnote%
\def\footnote{\protect\rmarkdownfootnote}

%%% Change title format to be more compact
\usepackage{titling}

% Create subtitle command for use in maketitle
\newcommand{\subtitle}[1]{
  \posttitle{
    \begin{center}\large#1\end{center}
    }
}

\setlength{\droptitle}{-2em}
  \title{}
  \pretitle{\vspace{\droptitle}}
  \posttitle{}
  \author{}
  \preauthor{}\postauthor{}
  \date{}
  \predate{}\postdate{}

\usepackage{float} \usepackage{lineno}
\usepackage{setspace}\doublespacing \usepackage[firstinits=true,style=authoryear-comp,natbib=true,doi=false,isbn=false,url=false,uniquename=false,uniquelist=false,sorting=none]{biblatex}

\usepackage{titlesec}

\titlespacing\section{0pt}{-5pt plus 1pt minus 1pt}{-5pt plus 1pt minus 1pt}
\titlespacing\subsection{0pt}{-5pt plus 1pt minus 1pt}{-5pt plus 1pt minus 1pt}
\titlespacing\subsubsection{0pt}{-5pt plus 1pt minus 1pt}{-5pt plus 1pt minus 1pt}

\begin{document}

Running head: Power to detect population trends {[}33 of 40
characters{]}

Title: Minimum time required to detect population trends: the need for
long-term monitoring programs \vspace{7 mm}

Author: \textsc{Easton R. White$^{1,2}$} \vspace{3 mm}

Address:\emph{
    \\$^1$Center for Population Biology \\
    University of California, Davis \\
    2320 Storer Hall \\
        University of California, Davis \\
        One Shields Avenue Davis, CA, USA} \vspace{3 mm}

\(^2\)Corresponding author:
\href{mailto:eawhite@ucdavis.edu}{\nolinkurl{eawhite@ucdavis.edu}} , 1
480 203 7931 \vspace{1 mm}

To be submitted to: \emph{Ecology} as a Statistical Report \vspace{1 mm}

Number of pages: 20\\
\vspace{1 mm}

Number of figures: 4 \vspace{1 mm}

Number of references: 33 \vspace{1 mm}

Supplementary material: 9 figures and 3 tables \vspace{2 mm}

Data and code for all the figures and tables can be found at
(\url{https://github.com/erwhite1/time-series-project}). \vspace{2 mm}

Available as a preprint at: \url{https://peerj.com/preprints/3168/}
\vspace{2 mm}

\pagebreak

\linenumbers

\begin{abstract}
Long-term time series are necessary to better understand population dynamics, assess species' conservation status, and make management decisions. However, population data are often expensive, requiring a lot of time and resources. When is a population time series long enough to address a question of interest? I determine the minimum time series length required to detect significant increases or decreases in population abundance. To address this question, I examine 822 populations of vertebrate species. I show that on average 15.9 years of continuous monitoring are required in order to achieve a high level of statistical power. However, there is a wide distribution around this average, casting doubt on simple rules of thumb. The minimum time required depends on trend strength, population variability, and temporal autocorrelation. However, there were no life-history traits (e.g. generation length) that were predictive of the minimum time required. These results point to the importance of sampling populations over long periods of time. I argue that statistical power needs to be considered in monitoring program design and evaluation. Short time series are likely under-powered and potentially misleading.
\end{abstract}

Keywords: ecological time series, experimental design, monitoring, power
analysis, statistical power, sampling design

\section{Introduction}\label{introduction}

Observational studies and population time series have become a
cornerstone of modern ecological research and conservation biology
(Magurran et al. 2010; Hughes et al. 2017). Long-term data are necessary
to both understand population dynamics and to assess species extinction
risk. Some time series may now be considered ``long-term''
(e.g.~continuous plankton recorder, Giron-Nava et al. (2017)), but most
are still short. Time series are typically short due to short funding
cycles and typical experimental time-frames (Field et al. 2007; Hughes
et al. 2017).

How long of a time series is actually necessary? This question has
important implications for both research and management (Nichols and
Williams 2006). A short time series may lead to wrong conclusions given
large natural year-to-year variability (McCain, Szewczyk, and Knight
2016). Managers need to know when action is needed for a population.
Therefore, managers must understand when a population trend over time is
actually meaningful. For example, the International Union for
Conservation of Nature (IUCN) Red List Categories and Criteria suggest,
under Criterion A2, a species qualifies as vulnerable if it has
experienced a 30\% decline over 10 years, or three generations (IUCN
2012). Sampling is typically expensive, therefore, we also do not want
to sample for longer than is necessary. For example, Gerber, DeMaster,
and Kareiva (1999) investigated the minimum time series required to
estimate population growth of the endangered, but recovering, eastern
North Pacific gray whale (\emph{Eschrichtius robustus}). They used a
long-term census to retroactively determine the minimum time series
required to assess threat status. They found that only 11 years were
needed, eight years before the delisting decision was made. This
highlights the importance of estimating the minimum time series required
as an earlier decision would have saved time and money (Gerber,
DeMaster, and Kareiva 1999). Further, waiting too long to decide an
action can imperil a species where management action could have been
taken earlier (Martin et al. 2012; Martin et al. 2017).

Past work has investigated questions related to the minimum time series
required to estimate trends in population size over time (Wagner,
Vandergoot, and Tyson 2009; Giron-Nava et al. 2017). For example, Rhodes
and Jonzen (2011) examined the optimal allocation of effort between
spatial and temporal replicates. Using simple populations models, they
found that the allocation of effort depends on environmental variation,
spatial and temporal autocorrelation, and observer error. Rueda-Cediel
et al. (2015) also used a modeling approach, but parameterized a model
specific for a threatened snail, \emph{Tasmaphena lamproides}. They
found that for this short-lived organism, 15 years was adequate to
assess long-term trends in abundance. However, these studies, and other
past work, have typically focused either on theoretical aspects of
monitoring design or focused on only a few species.

I estimate the minimum time series length required (\(T_{min}\)) to
assess long-term changes in abundance via simple linear regression.
Statistical power is not a new tool (Cohen 1992), but it is still
under-appreciated in ecological research (Legg and Nagy 2006).
Therefore, I only briefly describe the idea of statistical power for
time series analyses using simulations. I then examine 822 population
time series to estimate \(T_{min}\).

\section{Methods}\label{methods}

\subsection{Statistical power in time
series}\label{statistical-power-in-time-series}

An important step in experimental design is to determine the number of
samples required. For any particular experiment four quantities are
intricately linked: significance level (\(\alpha\)), statistical power,
effect size, and sample size (Legg and Nagy 2006). The exact
relationship between these quantities depends on the specific
statistical test. A type I error is a false positive, or incorrect
rejection of a true null hypothesis. For example, if a time series was
assessed as significantly increasing or decreasing---when there was no
true significant trend---this would be a false positive. The false
positive rate, or significance level (\(\alpha\)) is often set at 0.05
(although this is purely historical, Mapstone (1995)). A type II error
(\(\beta\)) is a failure to detect a true trend, or failure to reject a
false null hypothesis. Formally, statistical power (\(1-\beta\)) is one
minus the probability of a type II error (\(\beta\)). The effect size is
a estimate of the strength of a particular phenomenon. Prior to an
experiment, one could set appropriate levels of power, significance
level, and the effect size to estimate the sample size required for the
experiment. This approach, however, is not straight-forward for a time
series, or more complicated scenarios (P. C. Johnson et al. 2015), as
data are clearly non-independent.

\subsection{Simulation approach}\label{simulation-approach}

One approach to determining the minimum time series length needed is
through repetitive simulations of a population model (Gerrodette 1987).
This is the same approach one might use in sample size calculations for
any experimental design too complicated for simple power analyses
(Bolker 2008; P. C. Johnson et al. 2015). Essentially, a population
model, with a selected set of parameter values, is simulated
repetitively for a number of years. As an example, we can take the
following population model for population size \(N\) at time \(t\):

\begin{equation}
N(t + 1) = N(t) + r(t) + \epsilon \mbox{ with } \epsilon \sim N(\mu, \sigma)
\label{sim_model}
\end{equation}

where \(\epsilon\) is a normally-distributed random noise term with mean
\(\mu\) and standard deviation \(\sigma\). The rate of growth (\(r\)) is
the trend strength of the increase or decrease (i.e.~the estimated slope
from linear regression). It is important to note that any population
model could be used here. As an example, in the supplementary material
(Fig. \ref{fig:shark_example}), an age-structured model of population
dynamics is used in place of equation \ref{sim_model} (White, Nagy, and
Gruber 2014).

Statistical power is the proportion of simulations where the slope
parameter from linear regression is significant at the 0.05 threshold.
Statistical power of 0.8 would indicate that, if there was a true trend
in abundance, there would be a 0.8 probability of detecting the trend.
Values of 0.05 for the significance level and 0.8 statistical power are
historical (Cohen 1992) and it is important to examine the effect of
changing these values (Fig. \ref{fig:min_time_vs_alpha_beta}).

In Fig. \ref{fig:theoretical_approach}a, the model in equation
\ref{sim_model} is simulated repetitively. Statistical power increases
with increases in the length of time sampled (Fig.
\ref{fig:theoretical_approach}b). Where power is greater than 0.8 (the
dotted line), that is the minimum time required (\(T_{min}\)) to be
confident in the detection of a long-term trend in abundance. As shown
previously (Rhodes and Jonzen 2011; Rueda-Cediel et al. 2015),
statistical power increases with larger trend strength and lower
population variability (Fig. \ref{fig:theoretical_approach}c,d).
Simulation approaches can be useful before designing a monitoring
program or when a realistic model exists for the population in question.
However, it is also informative to examine how statistical power changes
with sampling effort in empirical time series.

\subsection{Data source}\label{data-source}

I use a database of 2444 population time series compiled by Keith et al.
(2015). The data are originally from the Global Population Dynamics
Database (NERC Centre for Population Biology 2010) and several other
sources (Keith et al. 2015). I filtered out short time series (less than
35 years), and those with missing data, leaving 822 time series. The
data includes information on 477 vertebrate species with a focus on
mammals, birds, and fish.

\subsection{Empirical approach}\label{empirical-approach}

I assume that each time series is long enough to include all necessary
information (e.g.~variability) about the population. In other words,
each time series is a representative sample. I first take all possible
contiguous subsamples of each time series. For example, a time series of
35 years would have 34 possible contiguous subsamples of length 2, 34
possible contiguous subsamples of length 3, and continuing until 1
possible contiguous subsample of length 35 (Gerber, DeMaster, and
Kareiva 1999; Brashares and Sam 2005; Giron-Nava et al. 2017). In line
with the simulation approach, I determine the proportion of subsamples
of a particular length that have estimated slope coefficients
statistically different from zero. This proportion is a measure of
statistical power. Lastly, I determine which subsample length is
required to achieve a certain threshold of statistical power (0.8, Cohen
(1992)). The minimum subsampled length that met these criteria is the
minimum time series length required (\(T_{min}\)). All analyses were
conducted in R (R Core Team 2016).

\section{Results}\label{results}

I examined a database of 822 separate population time series
representing 477 species. This database consisted of vertebrate species
with a variety of life-history characteristics. Across all the
populations, I found an average minimum time series length required
(\(T_{min}\)) of 15.9 (SD=8.3), with a wide distribution (Fig.
\ref{fig:min_time_dist}b). Estimates of \(T_{min}\) varied between
biological class (Fig. \ref{fig:min_time_dist}a). Ray-finned fish (class
Actinopterygii) typically had estimates of \(T_{min}\) over 20 years.
Birds (class Aves) had a much wider distribution of \(T_{min}\), but
usually required less years of sampling. Differences between these
classes can be explained by differences in variability in population
size and strength of trends in abundance (Fig. \ref{fig:class}).

\subsection{Corrrelates for minimum time
required}\label{corrrelates-for-minimum-time-required}

The minimum time series length required was strongly correlated with
trend strength (i.e.~estimated slope coefficient from linear
regression), coefficient of variation in population size, and
autocorrelation in population size (Figs. \ref{fig:correlates}a-c). All
three of these explanatory variables were significant and had large
effect sizes (see Table A1). Combined, trend strength, coefficient of
variation in population, and autocorrelation account for 75.1\% of the
explained deviance (Zuur et al. 2009) in minimum time series length
required.

There was life-history information available (Myhrvold et al. 2015) for
547 populations representing 315 different species, all of which were
birds (Aves class). Some life-history traits were significant predictors
for the minimum time required (Figs. \ref{fig:correlates}d-h, Tables
A2,A3). However, even combined, all five of the life-history traits
accounted for only 5.99\% of the explained deviance in minimum time
series length required. In addition, when accounting for trend strength,
coefficient of variation, and autocorrelation, no life-history traits
were significant predictors of the minimum time required (Table A3).

\subsection{Evalulating the IUCN
criteria}\label{evalulating-the-iucn-criteria}

I examined a subset of populations with observed declines of 30\% or
greater over 10 years, qualifying all of them as vulnerable under IUCN
Criterion A2 (IUCN 2012). This resulted in n = 162 populations. I then
compared the minimum time required to achieve 0.8 statistical power
(\(T_{min}\)) to the minimum time required under the IUCN criteria (Fig.
\ref{fig:IUCN_analysis}). For populations below the identity line in
figure \ref{fig:IUCN_analysis}, IUCN criteria would require more
sampling compared to estimates for \(T_{min}\). Further, populations
above the identity line are cases where the IUCN criteria would classify
a population as vulnerable despite not having sampled enough years to
achieve high statistical power (Fig. \ref{fig:IUCN_analysis}). The
silhouettes on figure \ref{fig:IUCN_analysis} highlight that species
with long generation times had larger discrepancies between \(T_{min}\)
and the minimum time required for IUCN assessments (Fig.
\ref{fig:IUCN_correlates}).

\section{Discussion}\label{discussion}

I examined 822 population time series. I then subsampled each to
determine the minimum time required to achieve a desired significance
level and power for linear regression. Statistical power is important as
it provides information on the necessary number of samples required to
determine a significant trend (Legg and Nagy 2006). I found that on
average 15.91 years of continuous monitoring were typically necessary
(Fig. \ref{fig:min_time_dist}b). However, the distribution of minimum
time required was wide. Therefore, simplistic rules of thumb are
probably not appropriate. This time-frame is in line with past work on a
short-lived snail species (Rueda-Cediel et al. 2015) and a long-lived
whale species (Gerber, DeMaster, and Kareiva 1999). Hatch (2003) used
seabird monitoring data to estimate minimum sampling requirements. He
found that the time required ranged from 11 to 69 years depending on
species, trend strength, and study design.

In line with theoretical predictions (Rhodes and Jonzen 2011), I also
found \(T_{min}\) was strongly correlated with the trend strength,
variability in population size, and temporal autocorrelation (Fig.
\ref{fig:correlates}). Contrary to my prior expectations, I also found
that \(T_{min}\) did not correlate with any life-history traits (Fig.
\ref{fig:correlates}d-h). I initially hypothesized that species with
longer lifespans or generation times may require a longer sampling
period. This result could have been a result of at least two factors.
First, the data I used may not include a diverse enough set of species
with different life-history traits. Second, the question I posed,
whether a population is increasing or decreasing, was specifically
concerned with population trends over time. Therefore, life-history
characteristics may be more important for other questions more closely
tied to species biology. For example, Blanchard, Maxwell, and Jennings
(2007) used detailed simulations of spatially-distributed fisheries to
compare survey designs. They found that statistical power depended on
survey design, temperature preferences, and the degree of population
patchiness.

An important related idea is the optimal allocation of sampling effort
in space versus time. In a theoretical investigation of this question,
Rhodes and Jonzen (2011) found that the optimal allocation of sampling
depended strongly on temporal and spatial autocorrelation. If spatial
population dynamics were highly correlated, then it was better to sample
more temporally, and vice versa. My work supports this idea as
populations with strong temporal autocorrelation needed less years of
sampling (Fig. \ref{fig:correlates}). Morrison and Hik (2008) also
studied the optimal allocation of sampling effort in space versus time,
but used emprical data from a long-term survey of the collared pika
(\emph{Ochotona collaris}) in the Yukon. They found that surveys less
than 5 years may be misleading and that extrapolating from one
population to another, even when nearby geographically, may be
untenable.

Seavy and Reynolds (2007) asked whether statistical power was even a
useful framework for assessing long-term population trends. They used 24
years of census data on Red-tailed Tropicbirds
(\emph{Phaethon rubricauda}) in Hawaii and showed that to detect a 50\%
decline over 10 years almost always resulted in high statistical power
(above 0.8). Therefore, they cautioned against only using power analyses
to design monitoring schemes and instead argued for metrics that would
increase precision: improving randomization, reducing bias, and
increasing detection probability. Power analyses should not be the only
consideration when designing monitoring schemes. However, unlike Seavy
and Reynolds (2007), the results here indicate that longer than 10 years
is often needed to achieve high statistical power.

As a test of the consequences of not conducting power analyses, I
compared the results here to the IUCN Redlist criteria. IUCN criteria A2
suggests that species that have experienced 30\% declines over 10 years
(or three generations) should be listed as vulnerable (IUCN 2012).
However, for the populations I examined, this criteria may be too
simplistic (Fig. \ref{fig:IUCN_analysis}). For many populations, the
IUCN criteria suggest more years than necessary are required to assess a
population as vulnerable (points below diagonal line in Fig.
\ref{fig:IUCN_analysis}). Conversely, for other populations the IUCN
criteria suggest sampling times that are less than the minimum time
required for statistical power. This suggests that the IUCN criteria are
probably too simplistic as the minimum time required does not correlate
with generation time or other biological covariates (Fig.
\ref{fig:correlates}d-h).

The design of monitoring programs should include calculations of
statistical power, the allocation of sampling in space versus time
(Rhodes and Jonzen 2011), and metrics to increase precision (Seavy and
Reynolds 2007). Ideally, a formal decision analysis to evaluate these
different factors would be conducted to design or assess any monitoring
program (Hauser, Pople, and Possingham 2006; McDonald-Madden et al.
2010). This type of formal decision analysis would also include
information on the costs of monitoring. These costs include the actual
costs of sampling (Brashares and Sam 2005) and the ecological costs on
inaction (Thompson et al. 2000).

\subsection{Limitations}\label{limitations}

This paper has some limitations in determining the minimum time series
length required. First, \(T_{min}\) is particular to the specific
question of interest. An additional complication is that for the
empirical approach, the subsampling of the full time series allows for
estimates of power, but the individual subsamples are clearly not
independent of one another. In an ideal setting, a specific population
model would be parameterized for each population of interest. Then,
model simulations could be used to estimate the minimum time series
required to address each specific question of interest. Clearly, this is
not always practical, especially if conducting analyses for a wide array
of species.

\subsection{Conclusions}\label{conclusions}

Power analyses are not a novel aspect of ecological research (Legg and
Nagy 2006). However, power analyses are still underutilized, especially
in the context of time series analyses. I used a database of 822
populations to determine the minimum time series length required to
detect population trends. This goes beyond previous work that either
focused on theoretical investigations or a limited number of species. I
show that to identify long-term changes in abundance, on average 15.91
years of continuous monitoring are often required (Fig.
\ref{fig:min_time_dist}). However, there is wide distribution of
estimated minimum times. Therefore, it is probably not wise to use a
simple threshold number of years in monitoring design. Further, contrary
to my initial hypotheses, minimum time required did not correlate with
generation time or any other life-history traits (Fig.
\ref{fig:correlates}d-h). This result argues against overly simplified
measures of minimum sampling time based on generation length or other
life-history traits (Fig. \ref{fig:IUCN_analysis}).

My work implies that for many populations, short time series are
probably not reliable for detecting population trends. This result
highlights the importance of long-term monitoring programs. From both a
scientific and management perspective, estimates of \(T_{min}\) are
important. If a time series is too short, we lack the statistical power
to reliably detect long-term population trends. In addition, a time
series that is too long may be a poor use of already limited funds
(Gerber, DeMaster, and Kareiva 1999). Further, more data is not always
best in situations where management actions need to be taken (Martin et
al. 2012; Martin et al. 2017). When a population trend is detected, it
may be too late for management action. In these situations, the
precautionary principle may be more appropriate (Thompson et al. 2000).
Future work should examine other species, with a wider range of
life-history characteristics. In addition, similar approaches can be
used to determine the minimum time series length required to address
additional questions of interest.

\section{Supporting Information}\label{supporting-information}

In the supporting material, I provide an expanded methods sections,
additional figures, minimum time calculations for determining
exponential growth, simulations with a more complicated population
model, and the use of generalized additive models to identify population
trends. All code and data can be found at
\url{https://github.com/erwhite1/time-series-project}

\section{Acknowledgements}\label{acknowledgements}

ERW was partially supported by a National Science Foundation Graduate
Fellowship. I would like to thank members of the Ecological Theory group
at the University of California, Davis for their insight.

\section{References}\label{references}

\hypertarget{refs}{}
\hypertarget{ref-Blanchard2007}{}
Blanchard, Julia L, David L Maxwell, and Simon Jennings. 2007. ``Power
of monitoring surveys to detect abundance trends in depleted
populations: the effects of density-dependent habitat use, patchiness,
and climate change.'' \emph{ICES Journal of Marine Science} 65 (1):
111--20.

\hypertarget{ref-Bolker2008}{}
Bolker, Benjamin M. 2008. \emph{Ecological Models and Data in R}. 1st
ed. Princeton, New Jersey: Princeton University Press.

\hypertarget{ref-Brashares2005}{}
Brashares, Justin S, and Moses K Sam. 2005. ``How much is enough?
Estimating the minimum sampling required for effective monitoring of
African reserves.'' \emph{Biodiversity and Conservation} 14: 2709--22.
doi:\href{https://doi.org/10.1007/s10531-005-8404-z}{10.1007/s10531-005-8404-z}.

\hypertarget{ref-Cohen1992}{}
Cohen, Jacob. 1992. ``A power primer.'' \emph{Psychological Bulletin}
112 (1): 155--59.
doi:\href{https://doi.org/10.1037/0033-2909.112.1.155}{10.1037/0033-2909.112.1.155}.

\hypertarget{ref-Field2007}{}
Field, Scott A, Patrick J O Connor, Andrew J Tyre, and Hugh P
Possingham. 2007. ``Making monitoring meaningful.'' \emph{Austral
Ecology} 32: 485--91.
doi:\href{https://doi.org/10.1111/j.1442-9993.2007.01715.x}{10.1111/j.1442-9993.2007.01715.x}.

\hypertarget{ref-Gerber1999}{}
Gerber, L R, D P DeMaster, and P M Kareiva. 1999. ``Gray whales and the
value of monitering data in implementing the U.S. endangered species
act.'' \emph{Conservation Biology} 13 (5): 1215--9.

\hypertarget{ref-Gerrodette1987}{}
Gerrodette, Tim. 1987. ``A power analysis for detecting trends.''
doi:\href{https://doi.org/10.2307/1939220}{10.2307/1939220}.

\hypertarget{ref-Giron-Nava2017}{}
Giron-Nava, Alfredo, Chase C James, Andrew F Johnson, David Dannecker,
Bethany Kolody, Adrienne Lee, Maitreyi Nagarkar, et al. 2017.
``Quantitative argument for long-term ecological monitoring.''
\emph{Marine Ecology Progress Series} 572: 269--74.

\hypertarget{ref-Hatch2003}{}
Hatch, S A. 2003. ``Statistical power for detecting trends with
applications to seabirds monitoring.'' \emph{Biological Conservation}
111: 317--29.

\hypertarget{ref-Hauser2006}{}
Hauser, Cindy E., Anthony R. Pople, and Hugh P. Possingham. 2006.
``Should managed populations be monitored every year?'' \emph{Ecological
Applications} 16 (2): 807--19.

\hypertarget{ref-Hughes2017}{}
Hughes, Brent B, Rodrigo Beas-luna, Allison K Barner, Kimberly Brewitt,
Daniel R Brumbaugh, Elizabeth B. Cerny-Chipman, Sarah L. Close, et al.
2017. ``Long-term studies contribute disproportionately to ecology and
policy.'' \emph{BioScience} 67 (3): 271--81.
doi:\href{https://doi.org/10.1093/biosci/biw185}{10.1093/biosci/biw185}.

\hypertarget{ref-IUCN2012}{}
IUCN. 2012. ``IUCN Red List Categories and Criteria: Version 3.1.''
doi:\href{https://doi.org/10.9782-8317-0633-5}{10.9782-8317-0633-5}.

\hypertarget{ref-Johnson2015}{}
Johnson, Paul CD, Sarah JE Barry, Heather M Ferguson, and Pie Müller.
2015. ``Power analysis for generalized linear mixed models in ecology
and evolution.'' \emph{Methods in Ecology and Evolution} 6 (2): 133--42.
doi:\href{https://doi.org/10.1111/2041-210X.12306}{10.1111/2041-210X.12306}.

\hypertarget{ref-Keith2015}{}
Keith, David, H. Resit Akçakaya, Stuart H.M. Butchart, Ben Collen,
Nicholas K. Dulvy, Elizabeth E. Holmes, Jeffrey A. Hutchings, et al.
2015. ``Temporal correlations in population trends: Conservation
implications from time-series analysis of diverse animal taxa.''
\emph{Biological Conservation} 192. Elsevier B.V.: 247--57.
doi:\href{https://doi.org/10.1016/j.biocon.2015.09.021}{10.1016/j.biocon.2015.09.021}.

\hypertarget{ref-Legg2006}{}
Legg, Colin J, and Laszlo Nagy. 2006. ``Why most conservation monitoring
is, but need not be, a waste of time.'' \emph{Journal of Environmental
Management} 78: 194--99.
doi:\href{https://doi.org/10.1016/j.jenvman.2005.04.016}{10.1016/j.jenvman.2005.04.016}.

\hypertarget{ref-Magurran2010}{}
Magurran, Anne E, Stephen R Baillie, Stephen T Buckland, Jan Mcp Dick,
David A Elston, E Marian Scott, Rognvald I Smith, Paul J Somerfield, and
Allan D Watt. 2010. ``Long-term datasets in biodiversity research and
monitoring : assessing change in ecological communities through time.''
\emph{Trends in Ecology and Evolution} 25: 574--82.
doi:\href{https://doi.org/10.1016/j.tree.2010.06.016}{10.1016/j.tree.2010.06.016}.

\hypertarget{ref-Mapstone1995}{}
Mapstone, Bruce D. 1995. ``Scalable decision rules for environmental
impact studies : effect Size , type I , and type II errors.''
\emph{Ecological Applications} 5 (2): 401--10.

\hypertarget{ref-Martin2017}{}
Martin, Tara G., Abbey E. Camaclang, Hugh P. Possingham, Lynn A.
Maguire, and Iadine Chadès. 2017. ``Timing of Protection of Critical
Habitat Matters.'' \emph{Conservation Letters} 10 (3): 308--16.
doi:\href{https://doi.org/10.1111/conl.12266}{10.1111/conl.12266}.

\hypertarget{ref-Martin2012}{}
Martin, Tara G., Simon Nally, Andrew A. Burbidge, Sophie Arnall, Stephen
T. Garnett, Matt W. Hayward, Linda F. Lumsden, Peter Menkhorst, Eve
Mcdonald-Madden, and Hugh P. Possingham. 2012. ``Acting fast helps avoid
extinction.'' \emph{Conservation Letters} 5 (4): 274--80.
doi:\href{https://doi.org/10.1111/j.1755-263X.2012.00239.x}{10.1111/j.1755-263X.2012.00239.x}.

\hypertarget{ref-McCain2016}{}
McCain, Christy Marie, Tim Szewczyk, and Kevin Bracy Knight. 2016.
``Population variability complicates the accurate detection of climate
change responses.'' \emph{Global Change Biology} 22 (6): 2081--93.
doi:\href{https://doi.org/10.1111/gcb.13211}{10.1111/gcb.13211}.

\hypertarget{ref-McDonald-Madden2010}{}
McDonald-Madden, Eve, Peter W J Baxter, Richard A. Fuller, Tara G.
Martin, Edward T. Game, Jensen Montambault, and Hugh P. Possingham.
2010. ``Monitoring does not always count.'' \emph{Trends in Ecology and
Evolution} 25 (10): 547--50.
doi:\href{https://doi.org/10.1016/j.tree.2010.07.002}{10.1016/j.tree.2010.07.002}.

\hypertarget{ref-Morrison2008}{}
Morrison, Shawm, and David S. Hik. 2008. ``When? Where? And for how
long? Census design considerations for an Alpine Lagomorph, the Collared
pika.'' In \emph{Lagomorph Biology}, 103--13. Springer Berlin
Heidelberg.
doi:\href{https://doi.org/10.1007/978-3-540-72446-9}{10.1007/978-3-540-72446-9}.

\hypertarget{ref-Myhrvold2015}{}
Myhrvold, Nathan P., Elita Baldridge, Benjamin Chan, Dhileep Sivam,
Daniel L. Freeman, and S.K. Morgan Ernest. 2015. ``An amniote
life-history database to perform comparative analyses with birds,
mammals, and reptiles.'' \emph{Ecology} 96 (11): 3109.

\hypertarget{ref-GPDD2010}{}
NERC Centre for Population Biology, Imperial College. 2010. ``The Global
Population Dynamics Database Version 2.''

\hypertarget{ref-Nichols2006}{}
Nichols, James D., and Bryon K. Williams. 2006. ``Monitoring for
conservation.'' \emph{Trends in Ecology and Evolution} 21 (12): 668--73.
doi:\href{https://doi.org/10.1016/j.tree.2006.08.007}{10.1016/j.tree.2006.08.007}.

\hypertarget{ref-RCoreTeam2016}{}
R Core Team. 2016. ``R: A language and environment for statistical
computing.'' Vienna, Austria: R Foundation for Statistical Computing.
\url{https://www.r-project.org/}.

\hypertarget{ref-Rhodes2011}{}
Rhodes, Jonathan R., and Niclas Jonzen. 2011. ``Monitoring temporal
trends in spatially structured populations: how should sampling effort
be allocated between space and time?'' \emph{Ecography} 34 (6): 1040--8.
doi:\href{https://doi.org/10.1111/j.1600-0587.2011.06370.x}{10.1111/j.1600-0587.2011.06370.x}.

\hypertarget{ref-Rueda-Cediel2015}{}
Rueda-Cediel, Pamela, Kurt E Anderson, Tracey J Regan, Janet Franklin,
and M Regan. 2015. ``Combined influences of model choice, data quality,
and data quantity when estimating population trends.'' \emph{PLoSONE} 10
(7): e0132255.
doi:\href{https://doi.org/10.1371/journal.pone.0132255}{10.1371/journal.pone.0132255}.

\hypertarget{ref-Seavy2007}{}
Seavy, Nathaniel E., and Michelle H. Reynolds. 2007. ``Is statistical
power to detect trends a good assessment of population monitoring?''
\emph{Biological Conservation} 140 (1-2): 187--91.
doi:\href{https://doi.org/10.1016/j.biocon.2007.08.007}{10.1016/j.biocon.2007.08.007}.

\hypertarget{ref-Thompson2000}{}
Thompson, Paul M, Ben Wilson, Kate Grellier, and Philip S Hammond. 2000.
``Combining power analysis and population viability analysis to compare
traditional and precautionary approaches to conservation of coastal
cetaceans.'' \emph{Conservation Biology} 14 (5): 1253--63.

\hypertarget{ref-Wagner2009}{}
Wagner, Tyler, Christopher S. Vandergoot, and Jeff Tyson. 2009.
``Evaluating the power to detect temporal trends in fishery-independent
surveys - A case study based on gill nets set in the Ohio waters of Lake
Erie for walleyes.'' \emph{North American Journal of Fisheries
Management} 29: 805--16.
doi:\href{https://doi.org/10.1577/M08-197.1}{10.1577/M08-197.1}.

\hypertarget{ref-White2014}{}
White, Easton R, John D Nagy, and Samuel H Gruber. 2014. ``Modeling the
population dynamics of lemon sharks.'' \emph{Biology Direct} 9 (23):
1--18.

\hypertarget{ref-Zuur2009}{}
Zuur, Alain F., Elena N. Ieno, Neil J. Walker, Anatoly A. Saveliev, and
Graham M. Smith. 2009. \emph{Mixed Effects Models and Extensions in
Ecology with R}. New York: Springer.

\clearpage

\pagebreak

\section{Figure captions}\label{figure-captions}

Figure 1: (a) Example of a simulated time series for 40 time periods.
(b) Statistical power versus the simulated time series length. The
horizontal, dashed line is the desired statistical power of 0.8. The
vertical, dashed line is the minimum time required to achieve the
desired statistical power. (c) Minimum time required (\(T_{min}\)) for
simulations with different values of the trend strength (\(r\)) and
\(\sigma = 5.0\). (d) Minimum time required for different levels of
population variability (\(\sigma\)) and \(r=1.5\). In each case, the
minimum time required is the minimum number of years to achieve 0.8
statistical power given a significance level of 0.05..

Figure 2: (a) Distributions of the minimum time required for populations
from four different biological classes. (b) Distribution of minimum time
required for all populations regardless of biological class. The minimum
time required calculation corresponds to a significance level of 0.05
and statistical power of 0.8.

Figure 3: Minimum time required to estimate change in abundance
correlated with (a) trend strength (absolute value of slope coefficient
estimated from linear regression), (b) coefficient of variation in
interannual population size, (c) temporal lag-1 autocorrelation, (d)
generation length (years), (e) litter size (n), (f) log adult body mass
(grams), (g) maximum longevity (years), and (h) incubation (days). The
lines in each plot represent the best fit line from linear regression.

Figure 4: Minimum time required to achieve 0.8 statistical power versus
the minimum time required under IUCN criteria A2 to classify a species
as vulnerable. Each point represents a single population, all of which
saw declines of 30\% or greater over a 10 year period. The silhouettes
highlight that species with longer generation times typically have
larger discrepancies between \(T_{min}\) and the minimum time required
for IUCN assessments.

\pagebreak

\begin{figure}
\centering
\includegraphics{manuscript_draft_short_version_files/figure-latex/unnamed-chunk-5-1.pdf}
\caption{(a) Example of a simulated time series for 40 time periods. (b)
Statistical power versus the simulated time series length. The
horizontal, dashed line is the desired statistical power of 0.8. The
vertical, dashed line is the minimum time required to achieve the
desired statistical power. (c) Minimum time required (\(T_{min}\)) for
simulations with different values of the trend strength (\(r\)) and
\(\sigma = 5.0\). (d) Minimum time required for different levels of
population variability (\(\sigma\)) and \(r=1.5\). In each case, the
minimum time required is the minimum number of years to achieve 0.8
statistical power given a significance level of
0.05.\label{fig:theoretical_approach}}
\end{figure}

\pagebreak

\begin{figure}
\centering
\includegraphics{manuscript_draft_short_version_files/figure-latex/unnamed-chunk-6-1.pdf}
\caption{(a) Distributions of the minimum time required for populations
from four different biological classes. (b) Distribution of minimum time
required for all populations regardless of biological class. The minimum
time required calculation corresponds to a significance level of 0.05
and statistical power of 0.8.\label{fig:min_time_dist}}
\end{figure}

\clearpage

\begin{figure}
\centering
\includegraphics{manuscript_draft_short_version_files/figure-latex/unnamed-chunk-9-1.pdf}
\caption{Minimum time required to estimate change in abundance
correlated with (a) trend strength (absolute value of slope coefficient
estimated from linear regression), (b) coefficient of variation in
interannual population size, (c) temporal lag-1 autocorrelation, (d)
generation length (years), (e) litter size (n), (f) log adult body mass
(grams), (g) maximum longevity (years), and (h) incubation (days). The
lines in each plot represent the best fit line from linear
regression.\label{fig:correlates}}
\end{figure}

\clearpage

\begin{figure}
\centering
\includegraphics{manuscript_draft_short_version_files/figure-latex/unnamed-chunk-10-1.pdf}
\caption{Minimum time required to achieve 0.8 statistical power versus
the minimum time required under IUCN criteria A2 to classify a species
as vulnerable. Each point represents a single population, all of which
saw declines of 30\% or greater over a 10 year period. The silhouettes
highlight that species with longer generation times typically have
larger discrepancies between \(T_{min}\) and the minimum time required
for IUCN assessments.\label{fig:IUCN_analysis}}
\end{figure}

\clearpage

\section{Supplementary material}\label{supplementary-material}

\setcounter{figure}{0} \renewcommand{\thefigure}{A\arabic{figure}}

\begin{figure}[htbp]
\centering
\caption{(a) Population size of Bigeye tuna (\emph{Thunnus obesus}) over
time. The line is the best fit line from linear regression. (b)
Statistical power for different subsets of the time series in panel
a.\label{fig:empirical_approach_example}}
\end{figure}

\begin{figure}[htbp]
\centering
\caption{Output of generalized linear model with a Poisson error
structure for predicting the minimum time required with explanatory
variables of the absolute value of the slope coefficient (or trend
strength), temporal autocorrelation, and variability in population
size.\label{fig:poisson_model}}
\end{figure}

\begin{figure}[htbp]
\centering
\caption{(a) Minimum time required to estimate change in abundance by biological class, (b) long-term trend (estimated slope coefficient) by class, (c) coefficient of variation in population size by class, and (d) temporal autocorrelation by class.\label{fig:class}}
\end{figure}

\begin{figure}[htbp]
\centering
\caption{Minimum time required to assess long-term trends in abundance
for values of statistical significance (\(\alpha\)) and power
(\(1-\beta\)).\label{fig:min_time_vs_alpha_beta}}
\end{figure}

\begin{figure}[htbp]
\centering
\caption{The difference between minimum time estimates is the minimum time required to achieve 0.8 statistical power versus the minimum time required under IUCN criteria A2 to classify a species as vulnerable. Each point represents a single population, all of which saw declines of 30\% or greater over a 10 year period. (a) Difference between minimum time estimates versus the coefficient of variation in population size. (b) Difference between minimum time estimates versus the generation length in years.\label{fig:IUCN_correlates}}
\end{figure}

\begin{figure}[htbp]
\centering
\caption{Distribution of the minimum time required in order to detect a
significant trend (at the 0.05 level) in log(abundance) given power of
0.8.\label{fig:min_time_dist_log_pop}}
\end{figure}

\begin{figure}[htbp]
\centering
\caption{Statistical power for different length of time series
simulations for a lemon shark population in Bimini,
Bahamas.\label{fig:shark_example}}
\end{figure}

\begin{figure}[htbp]
\centering
\caption{(a) Time series for Bigeye tuna (\emph{Thunnus obesus}) with
corresponding fitted GAM model in red and (b) statistical power as a
function of the number of years sampled. The horizontal line at 0.8
indicates the minimum threshold for statistical power and the vertical
line denotes the minimum time required to achieve 0.8 statistical
power.\label{fig:gam_example}}
\end{figure}

\begin{figure}[htbp]
\centering
\caption{Distribution of the minimum time required in order to detect a
significant trend (at the 0.05 level) in abundance according to a GAM
model given statistical power of 0.8. The smoothing parameter was set to
3 for each population.\label{fig:min_time_dist_gam}}
\end{figure}

\vspace{1 cm}

Table A1: Output of generalized linear model to examine time series
characteristics as correlates of the minimum time required for
determining long-term population trends.

\vspace{1 cm}

Table A2: Output of generalized linear model to examine life-history
trait correlates of the minimum time required for determine long-term
population trends.

\vspace{1 cm}

Table A3: Output of generalized linear model to examine both time series
characteristics and life-history trait correlates of the minimum time
required for determine long-term population trends.


\end{document}
