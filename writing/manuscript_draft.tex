\documentclass[12pt,]{article}
\usepackage{lmodern}
\usepackage{amssymb,amsmath}
\usepackage{ifxetex,ifluatex}
\usepackage{fixltx2e} % provides \textsubscript
\ifnum 0\ifxetex 1\fi\ifluatex 1\fi=0 % if pdftex
  \usepackage[T1]{fontenc}
  \usepackage[utf8]{inputenc}
\else % if luatex or xelatex
  \ifxetex
    \usepackage{mathspec}
  \else
    \usepackage{fontspec}
  \fi
  \defaultfontfeatures{Ligatures=TeX,Scale=MatchLowercase}
\fi
% use upquote if available, for straight quotes in verbatim environments
\IfFileExists{upquote.sty}{\usepackage{upquote}}{}
% use microtype if available
\IfFileExists{microtype.sty}{%
\usepackage{microtype}
\UseMicrotypeSet[protrusion]{basicmath} % disable protrusion for tt fonts
}{}
\usepackage[margin=1in]{geometry}
\usepackage{hyperref}
\hypersetup{unicode=true,
            pdfborder={0 0 0},
            breaklinks=true}
\urlstyle{same}  % don't use monospace font for urls
\usepackage{graphicx,grffile}
\makeatletter
\def\maxwidth{\ifdim\Gin@nat@width>\linewidth\linewidth\else\Gin@nat@width\fi}
\def\maxheight{\ifdim\Gin@nat@height>\textheight\textheight\else\Gin@nat@height\fi}
\makeatother
% Scale images if necessary, so that they will not overflow the page
% margins by default, and it is still possible to overwrite the defaults
% using explicit options in \includegraphics[width, height, ...]{}
\setkeys{Gin}{width=\maxwidth,height=\maxheight,keepaspectratio}
\IfFileExists{parskip.sty}{%
\usepackage{parskip}
}{% else
\setlength{\parindent}{0pt}
\setlength{\parskip}{6pt plus 2pt minus 1pt}
}
\setlength{\emergencystretch}{3em}  % prevent overfull lines
\providecommand{\tightlist}{%
  \setlength{\itemsep}{0pt}\setlength{\parskip}{0pt}}
\setcounter{secnumdepth}{5}
% Redefines (sub)paragraphs to behave more like sections
\ifx\paragraph\undefined\else
\let\oldparagraph\paragraph
\renewcommand{\paragraph}[1]{\oldparagraph{#1}\mbox{}}
\fi
\ifx\subparagraph\undefined\else
\let\oldsubparagraph\subparagraph
\renewcommand{\subparagraph}[1]{\oldsubparagraph{#1}\mbox{}}
\fi

%%% Use protect on footnotes to avoid problems with footnotes in titles
\let\rmarkdownfootnote\footnote%
\def\footnote{\protect\rmarkdownfootnote}

%%% Change title format to be more compact
\usepackage{titling}

% Create subtitle command for use in maketitle
\newcommand{\subtitle}[1]{
  \posttitle{
    \begin{center}\large#1\end{center}
    }
}

\setlength{\droptitle}{-2em}
  \title{}
  \pretitle{\vspace{\droptitle}}
  \posttitle{}
  \author{}
  \preauthor{}\postauthor{}
  \date{}
  \predate{}\postdate{}

\usepackage{float} \usepackage{lineno}
\usepackage[firstinits=true,style=authoryear-comp,natbib=true,doi=false,isbn=false,url=false,uniquename=false,uniquelist=false,sorting=none]{biblatex}
\linenumbers

\begin{document}

Title: Minimum time required to detect population trends: the need for
long-term monitoring programs \vspace{7 mm}

Author: \textsc{Easton R. White$^{1,2}$} \vspace{3 mm}

Address:\emph{
    \\$^1$Center for Population Biology \\
    University of California, Davis \\
    2320 Storer Hall \\
        University of California, Davis \\
        One Shields Avenue Davis, CA, USA} \vspace{3 mm}

\(^2\)Corresponding author:
\href{mailto:eawhite@ucdavis.edu}{\nolinkurl{eawhite@ucdavis.edu}}
\vspace{3 mm}

Number of words (including 300 words per figure): 5900 \vspace{1 mm}

Number of figures: 3 \vspace{1 mm}

Supplementary material: 11 figures and 1 table \vspace{3 mm}

Data and code for all the figures and tables can be found at
(\url{https://github.com/erwhite1/time-series-project}).

\vspace{3 mm}

\pagebreak 

\linenumbers

\begin{abstract}
Long-term time series are necessary to better understand population dynamics, assess species' conservation status, and make management decisions. However, population data are often expensive, requiring a lot of time and resources. When is a population time series long enough to address a question of interest? We determine the minimum time series length required to detect significant increases or decreases in population abundance. To address this question, we use simulation methods and examine 878 populations of vertebrate species. Here we show that 15-20 years of continuous monitoring are required in order to achieve a high level of statistical power. For both simulations and the time series data, the minimum time required depends on trend strength, population variability, and temporal autocorrelation. These results point to the importance of sampling populations over long periods of time. We argue that statistical power needs to be considered in monitoring program design and evaluation. Time series less than 15-20 years are likely underpowered and potentially misleading.
\end{abstract}

Keywords: ecological time series, experimental design, monitoring, power
analysis, statistical power, sampling design

\section{Introduction}\label{introduction}

Observational studies and population time series have become a
cornerstone of modern ecological research and conservation biology
(Magurran et al. 2010; Hughes et al. 2017). Long-term data are necessary
to both understand population dynamics and to assess species extinction
risk. Even though many time series may now be considered ``long-term''
(e.g.~continuous plankton recorder, Giron-Nava et al. (2017)), most are
still short. Time series are typically short for a variety of reasons
(Field et al. 2007). They are often coupled with an experiment, which
may only last a couple of years. In addition, short funding cycles make
it difficult to examine populations over longer periods of time (Hughes
et al. 2017).

How long of a time series is actually necessary? This question has
important implications for both research and management (Nichols and
Williams 2006). Scientists need to know the time series length required
to address a specific question. A short time series may lead to wrong
conclusions given large natural year-to-year variability. Managers need
to know when action is needed for a population. Therefore, they must
understand when population trend over time is actually meaningful. The
IUCN Red List Categories and Criteria suggest, under Criterion A2, a
species qualifies as vulnerable if it has experienced a 30\% decline
over 10 years, or 3 generations (IUCN 2012). For both scientific and
management questions, because sampling is typically expensive, we also
do not want to sample for longer than is necessary. For example, Gerber,
DeMaster, and Kareiva (1999) investigated the minimum time series
required to estimate population growth of the endangered, but
recovering, eastern North Pacific gray whale
(\emph{Eschrichtius robustus}). They used a long-term census to
retroactively determine the minimum time series required to assess
threat status. They found that only 11 years were needed, eight years
before the delisting decision was made. This highlights the importance
of estimating the minimum time series required as an earlier decision
would have saved time and money (Gerber, DeMaster, and Kareiva 1999).

An important step in experimental design is to determine the number of
samples required. For any particular experiment four quantities are
intricately linked: significance level (\(\alpha\)), statistical power,
effect size, and sample size (Legg and Nagy 2006). The exact
relationship between these quantities depends on the specific
statistical test. A type I error is a false positive, or incorrect
rejection of a true null hypothesis. For example, if a time series was
assessed as significantly increasing or decreasing---when there was no
true significant trend---this would be a false positive. The false
positive rate, or significance level (\(\alpha\)) is often set at 0.05
(although this is purely historical, Mapstone (1995)). A type II error
(\(\beta\)) is a failure to detect a true trend, or failure to reject a
false null hypothesis. Formally, statistical power (\(1-\beta\)) is one
minus the probability of a type II error (\(\beta\)). The effect size is
a measure of the difference between two groups. Prior to an experiment,
one could set appropriate levels of power, significance level, and the
effect size to estimate the sample size required for the experiment.
This approach, however, is not straight-forward for a time series, or
more complicated scenarios (P. C. Johnson et al. 2015), as data are
clearly non-independent.

For time series data, two general approaches to estimating sample size
are appropriate. Simulations can be designed for a specific population
and question (Bolker 2008; P. C. Johnson et al. 2015). Simple models can
be simulated with parameter values corresponding to a population of
interest (Gerrodette 1987). Statistical power can then be calculated as
the fraction of simulations that meet some criteria. The specific
criteria depend on the question at hand. For example, given a time
series, when is the slope from linear regression significantly different
from zero? In other words, when is the time series significantly
increasing or decreasing? It is then possible to determine how power
changes with a variable of interest. For example, time series can be
simulated for different lengths of time. From these simulations, the
minimum time series length required is calculated to meet certain levels
of statistical significance and power (Bolker 2008).

In addition to using simulations, empirical time series can also be
used. Multiple replicates of similar populations are usually not
available, but it is possible to subsample an empirical time series
(Gerber, DeMaster, and Kareiva 1999). Subsamples of different lengths
can then be evaluated to see which fraction of subsamples meet some
criteria, again a measure of statistical power. Similar to the
simulation approach, these measures of power can be used to determine
the minimum time series required for a particular question of interest.

Past work has investigated questions related to the minimum time series
required to estimate trends in population size over time (Wagner,
Vandergoot, and Tyson 2009; Giron-Nava et al. 2017). For example, Rhodes
and Jonzen (2011) examined the optimal allocation of effort between
spatial and temporal replicates. Using simple populations models, they
found that the allocation of effort depends on environmental variation,
spatial and temporal autocorrelation, and observer error. Rueda-Cediel
et al. (2015) also used a modeling approach, but parameterized a model
specifically for a threatened snail, \emph{Tasmaphena lamproides}. They
found that for this short-lived organism, 15 years was adequate to
assess long-term trends in abundance. Hatch (2003) examined seabird
populations and determined the minimum time required to achieve high
statistical power ranged between 11 and 69 years. However, these
studies, and other past work, have typically focused either on
theoretical aspects of monitoring design or focused on only a few
species.

We use both simulations and empirical time series to determine the
minimum number of years required to address several questions. We
estimate the minimum time series length required (\(T_{min}\)) to assess
long-term changes in abundance via simple linear regression. First, we
estimate \(T_{min}\) using a simulation approach. Then we examine 878
population time series to estimate \(T_{min}\). In the supplementary
material, we determine \(T_{min}\) for related ideas: calculating
long-term growth rates, using more complicated population models,
varying statistical level and power, and the use of generalized additive
models.

\section{Methods}\label{methods}

\subsection{Simulation approach}\label{simulation-approach}

One approach to determining the minimum time series length needed is
through repetitive simulations of a population model (Gerrodette 1987).
This is the same approach one might use in sample size calculations for
any experimental design too complicated for simple power analyses
(Bolker 2008; P. C. Johnson et al. 2015). We only briefly discuss this
approach as it has been described elsewhere. Essentially, we use a
population model and repetitively simulate it for a number of years.
This approach requires us to determine values for our model parameters
(e.g.~population variability). As an example, we can take the following
population model for population size \(N\) at time \(t\):

\begin{equation}
N(t + 1) = N(t) + r(t) + \epsilon \mbox{ with } \epsilon \sim N(\mu, \sigma)
\label{sim_model}
\end{equation}

where \(\epsilon\) is a normally-distributed random noise term with mean
\(\mu\) and standard deviation \(\sigma\). The rate of growth \(r\) is
also the trend strength of the increase or decrease (i.e.~the rate of
increase). It is important to note that any population model could be
substituted for equation \ref{sim_model}, as we do in the supplementary
material (Figs. \ref{fig:min_time_dist_log_pop},
\ref{fig:shark_example}).

Statistical power is then the fraction of simulations that meet some
criteria. Here, our criteria is whether the slope parameter from linear
regression is significant at the 0.05 threshold. Statistical power of
0.8 would indicate that, if there was a true trend in abundance, we
would have a 0.8 probability of detecting the trend. We also tested the
effect of varying both the significance level and statistical power
(Fig. \ref{fig:min_time_vs_alpha_beta}).

In Fig. \ref{fig:theoretical_approach}a, a number of simulated time
series are shown for a set number of time periods (\(t=40\)). It is
clear that statistical power increases quickly with increases in length
of time sampled (Fig. \ref{fig:theoretical_approach}b). Where power is
greater than 0.8 (the dotted line), that is the minimum time required
(\(T_{min}\)).

\begin{figure}[htbp]
\centering
\includegraphics{manuscript_draft_files/figure-latex/unnamed-chunk-1-1.pdf}
\caption{(a) Example of a simulated time series for 40 time periods. (b)
Statistical power versus the simulated time series length. The
horizontal, dashed line is the desired statistical power of 0.8. The
vertical, dashed line is the minimum time required to achieve the
desired statistical power. (c) Minimum time required (\(T_{min}\)) for
simulations with different values of the trend strength (\(r\)). (d)
Minimum time required for different levels of population variability
(\(\sigma\)). In each case, the minimum time required is the minimum
number of years to achieve 0.8 statistical power given a significance
level of 0.05.\label{fig:theoretical_approach}}
\end{figure}

\subsection{Data source}\label{data-source}

We use a database of 2444 population time series compiled in (Keith et
al. 2015); they compared the predictability of growth rates among
populations. The data are originally from the Global Population Dynamics
Database (NERC Centre for Population Biology 2010). We filtered out
short time series (less than 35 years), and those with missing data, and
were left with 878 time series. The data includes information on 478
vertebrate species with a focus on mammals, birds, and fish. The data
also includes information on generation length and census
specifications. For each time series, we also calculated variables of
interest like variance in population size, long-term trend in abundance
(slope coefficient from simple linear regression), and temporal
autocorrelation. All analyses were conducted in R (R Core Team 2016).

For a subset of populations (\(n =\) 547), we had information on
biological characteristics from another paper (Myhrvold et al. 2015),
including body size and generation length. All 547 populations were
birds. We examine how the minimum time required is related to these
biological characteristics (Fig. \ref{fig:biological_correlates}).

\subsection{Empirical approach}\label{empirical-approach}

We assume that each time series in our database is long enough to
include all necessary information (e.g.~variability) about the
population. In other words, each time series is a representative sample.
We first take all possible contiguous subsamples of each time series.
For example, a time series of 35 years would have 34 possible contiguous
subsamples of length 2, 34 possible contiguous subsamples of length 3,
and continuing until 1 possible contiguous subsample of length 35
(Gerber, DeMaster, and Kareiva 1999; Giron-Nava et al. 2017). Next, we
run a linear regression for each subsampled time series. Then, we
determine the fraction of subsamples of a particular length that have
estimated slope coefficients which are statistically different from
zero. We only look at the fraction of samples where the long-term, or
``true'', time series also has a significant slope. This fraction is a
measure of statistical power. Lastly, we determine which subsample
length is required to achieve a certain threshold of statistical power
(0.8, Cohen (1992)). The subsampled length meeting these criteria is the
minimum time series length required (\(T_{min}\)).

In the supplementary material, we show how the same approach described
here can be used to determine the minimum time required to estimate a
population's geometric growth rate (Figs.
\ref{fig:growth_rate},\ref{fig:min_time_growth_dist}). We also determine
the minimum time required to estimate long-term trends according to
generalized additive models, instead of the simple linear models used
here (Fig. \ref{fig:gam_example}).

\section{Results}\label{results}

We determined the minimum time series length (\(T_{min}\)) required to
address a particular question of interest. What is the minimum time
series length required to determine, via linear regression, the
long-term population trend? Here, the minimum time series length
required had high enough statistical power (\(1-\beta\) greater than
0.8) for a set significance level (\(\alpha\)) of 0.05. It is also
possible to alter \(1-\beta\) and \(\alpha\). Predictably, as we
increased our level of power or decreased \(\alpha\), \(T_{min}\)
increased (Fig. \ref{fig:min_time_vs_alpha_beta}). We estimated
\(T_{min}\) using two approaches. We briefly describe results from the
simulation approach and then discuss our empirical approach.

\subsection{Simulation approach}\label{simulation-approach-1}

We constructed a general population model where the trend strength
(i.e.~slope coefficient) over time could be a model parameter. We then
ran simulated time series of different lengths. From these simulations
we determined the minimum time series length required to achieve a
certain level of statistical power. In line with past work (Gerrodette
1987), we found the \(T_{min}\) increases (i.e.~more time is required)
with decreases in trend strength and with increases in population
variability (Figs. \ref{fig:theoretical_approach}c,d).

We chose a simple model, but any other population model could be used
(see example in Fig. \ref{fig:min_time_dist_log_pop}). Ideally, the
specific model choice should be tailored to the population of interest.
We explored how the simulation approach can be applied to more
biologically-realistic population models (Fig. \ref{fig:shark_example}).
More specifically, we determined the minimum time required to estimate
long-term population trends using a stochastic, age-structured model of
lemon shark population dynamics in the Bahamas (White, Nagy, and Gruber
2014). We found that over 27 years of continuous monitoring were needed
in this particular scenario (Fig. \ref{fig:shark_example}). Similar to
the simulation approach described above, the minimum time required for
the lemon shark population was strongly dependent on model parameters.

\subsection{Empirical approach}\label{empirical-approach-1}

\begin{figure}[htbp]
\centering
\includegraphics{manuscript_draft_files/figure-latex/unnamed-chunk-4-1.pdf}
\caption{Distributions of the minimum time required for four different
biological classes. The minimum time required calculation corresponds to
a significance level of 0.05 and statistical power of
0.8.\label{fig:min_time_dist}}
\end{figure}

\begin{figure}[htbp]
\centering
\includegraphics{manuscript_draft_files/figure-latex/unnamed-chunk-5-1.pdf}
\caption{Minimum time required to estimate change in abundance
correlated with (a) trend strength (absolute value of slope coefficient
estimated from linear regression), (b) population variance (inter-annual
variability in population size), and (c) temporal lag-1
autocorrelation.\label{fig:correlates}}
\end{figure}

We examined a database of 878 separate population time series
representing 478 species. This database consists of vertebrate species
with a variety of life history characteristics (Fig.
\ref{fig:biological_correlates}). We limited our analyses to populations
with at least 35 years of continuous sampling. We then examined the
minimum time required to estimate long-term trends via linear
regression.

Across all the populations we examined, we found an average minimum time
series length required (\(T_{min}\)) of 16.5 (\(\sigma\)=8.4), with a
wide distribution (Fig. \ref{fig:min_time_dist}). Estimates of
\(T_{min}\) varied between biological class (Fig.
\ref{fig:min_time_dist}). Ray-finned fish (class Actinopterygii)
typically had estimates of \(T_{min}\) over 20 years. Birds (class Aves)
had a much wider distribution of \(T_{min}\), but usually required less
years of sampling. Differences between these classes were explained by
differences in inter-annual variability in population size (Fig.
\ref{fig:class}). We also examined a subset of populations where
life-history characteristics were known. None of these explanatory
variables were predictive of the minimum time series length required
(Fig. \ref{fig:biological_correlates}).

The minimum time series length required was strongly correlated with
trend strength, population variance, and autocorrelation (Fig.
\ref{fig:correlates}). This is in line based on our simulations and
those of others (Rhodes and Jonzen 2011). Using a generalized linear
model, with a Poisson error structure, all three of these explanatory
variables were significant and had large effect sizes (see Table A1).
Combined, trend strength, population variance, and autocorrelation
account for about 72.6\% of the explained deviance (Zuur et al. 2009) in
minimum time series length required.

Lastly, we tested model sensitivity by using generalized additive models
(GAMs) instead of simple linear regression. Again, we examined the
minimum time required to estimate long-term population trends (Fig.
\ref{fig:gam_example}). We found that although we obtain a similar
distribution of minimum times required for GAMs, the minimum time
required for GAMs is on average shorter than for linear regression (Fig.
\ref{fig:min_time_dist_gam}).

\section{Discussion}\label{discussion}

We explored two approaches to estimate the minimum time series length
required to address a particular question of interest. We asked, what is
the minimum time series length required to determine long-term
population trends using linear regression? This is one of the simplest
questions one could ask of a time series. The simulation-based approach
has been suggested by others, especially in situations more complicated
than that suited for classic power analysis (Gerrodette 1987; P. C.
Johnson et al. 2015; Bolker 2008). Our simulations support past work
that longer time series are needed when the trend strength (i.e.~rate of
increase or decrease) is weak or when population variability is high
(Gerrodette 1987). We also showed how the simulation model can be
altered for a particular population (Fig. \ref{fig:shark_example}) or
question (Figs. \ref{fig:min_time_dist_log_pop},\ref{fig:gam_example}).

Here, we focus on an empirical approach to estimate the minimum time
series length required to assess changes in abundance over time. We
examined 878 population time series (all longer than 35 years). We then
subsampled each to determine the minimum time required to achieve a
desired significance level and power for linear regression. Statistical
power is important as it provides on information as to the necessary
samples required to determine a significant trend (Legg and Nagy 2006).
We found that at least 15-20 years of continuous monitoring were
typically necessary (Fig. \ref{fig:min_time_dist}). This time frame is
in line with past work on a short-lived snail species (Rueda-Cediel et
al. 2015) and a long-lived whale species (Gerber, DeMaster, and Kareiva
1999). Hatch (2003) used seabird monitoring data to estimate minimum
sampling requirements. He found that the time required ranged from 11 to
69 years depending on species, trend strength, and study design.

In line with theoretical predictions (Rhodes and Jonzen 2011), we also
found \(T_{min}\) was strongly correlated with the trend strength,
variability in population size, and temporal autocorrelation (Fig.
\ref{fig:correlates}). Contrary to our prior expectations, we also found
that \(T_{min}\) did not correlate with any biological variables of
interest (Fig. \ref{fig:biological_correlates}). We initially
hypothesized that species with longer lifespans or generation times may
require a longer sampling period. Our result could have been a result of
at least two factors. First, the data we used may not include a diverse
enough set of species with different life history traits. Second, the
question we poised, whether a population is increasing or decreasing,
was specifically concerned with trends in population density over time.
Therefore, life-history characteristics may be more important for other
questions, like estimating species extinction risk (J. A. Hutchings et
al. 2012).

An important related question, is the optimal allocation of sampling
effort in space versus time. In a theoretical investigation of this
question, Rhodes and Jonzen (2011) found that the optimal allocation of
sampling depended strongly on temporal and spatial autocorrelation. If
spatial population dynamics were highly correlated, then it was better
to sample more temporally, and vice versa. Our work supports this idea
as populations with strong temporal autocorrelation needed less years of
sampling (Fig. \ref{fig:correlates}). Morrison and Hik (2008) also
studied the optimal allocation of sampling effort in space versus time,
but used emprical data from a long-term census of the collared pika
(\emph{Ochotona collaris}) found in the Yukon. They estimated long-term
growth rates among three subpopulations over a 10-year period. They
found that censuses less than 5 years may be misleading and that
extrapolating from one population to another, even when nearby
geographically, may be untenable.

Seavy and Reynolds (2007) asked whether statistical power was even a
useful framework for assessing long-term population trends. They used 24
years of census data on Red-tailed Tropicbirds
(\emph{Phaethon rubricauda}) in Hawaii and showed that to detect a 50\%
decline over 10 years almost always resulting in high statistical power
(above 0.8). Therefore, they cautioned against only using power analyses
to design monitoring schemes and instead argued for metrics that would
increase precision. For example, Seavy and Reynolds (2007) suggest
improving randomization, reducing bias, and increases detection
probability when designing and evaluating monitoring programs. We agree
that power analyses should not be the only consideration when designing
monitoring schemes. However, unlike Seavy and Reynolds (2007), our
results indicate that longer than 10 years is often needed to achieve
high statistical power. Therefore, the design of monitoring programs
should include calculations of statistical power, the allocation of
sampling in space versus time (Rhodes and Jonzen 2011), and metrics to
increase precision. Ideally, a formal decision analysis to evaluate
these different factors would be conducted to design or assess any
monitoring program (Hauser, Pople, and Possingham 2006; McDonald-Madden
et al. 2010).

\subsection{Limitations}\label{limitations}

Our work has some limitations in determining the minimum time series
length required. First, \(T_{min}\) is particular to the specific
question of interest. An additional complication is that for our
empirical approach, the subsampling of the full time series allows for
estimates of power, but the individual subsamples are clearly not
independent of one another. Further, estimates of \(T_{min}\) depend on
chosen values of \(\alpha\) and \(\beta\) (Fig.
\ref{fig:min_time_vs_alpha_beta}). In an ideal setting, we would build a
specific population model parameterized for each population of interest.
Then, model simulations could be used to estimate the minimum time
series required to address each specific question of interest. Clearly,
this is not always practical, especially if conducting analyses for a
wide array of species as we do here. In addition, our statistical models
suggest that \(T_{min}\) does not correlate with any biological
variables of interest, at least for the question of linear regression
(Fig. \ref{fig:biological_correlates}). Therefore, it is not possible to
use these results to predict \(T_{min}\) for another population, even if
the population is of a species with a similar life-history to one in our
database.

\subsection{Conclusions}\label{conclusions}

We use a database of 878 populations to determine the minimum time
series length required to detect population trends. This goes beyond
previous work that either focused on theoretical investigations or a
limited number of species. We show that to identify long-term changes in
abundance, 15-20 years of continuous monitoring are often required (Fig.
\ref{fig:min_time_dist}). In line with theoretical predictions
(Gerrodette 1987), we also show that \(T_{min}\) is strongly correlated
with the long-term population trend (i.e.~rate of increase), variability
in population size, and the temporal autocorrelation (Fig.
\ref{fig:correlates}). Our work implies that for many populations, time
series less than 15-20 years are probably not reliable for detecting
population trends. This result highlights the importance of long-term
monitoring programs. From both a scientific and management perspective
estimates of \(T_{min}\) are important. If a time series is too short,
we lack statistical power to reliably detect long-term population
trends. In addition, a time series that is too long may be a poor use of
already limited funds (Gerber, DeMaster, and Kareiva 1999). Future work
should examine other species, with a wider range of life history
characteristics. In addition, similar approaches can be used to
determine the minimum time series length required to address additional
questions of interest.

\section{Supporting Information}\label{supporting-information}

In the supporting material, we provide an expanded methods sections,
additional figures, minimum time calculations for determining
exponential growth, simulations with a more complicated population
model, minimum time calculations for determining long-term growth rates,
and the use of generalized additive models to identify population
trends. All code and data can be found at
\url{https://github.com/erwhite1/time-series-project}

\section{Acknowledgements}\label{acknowledgements}

ERW was partially supported by a National Science Foundation Graduate
Fellowship. We would like to thank members of the Ecological Theory
group at the University of California, Davis for their insight. We would
also like to thank T. Dallas and E. Malcolm-White for their helpful
comments.

\section{References}\label{references}

\hypertarget{refs}{}
\hypertarget{ref-Bolker2008}{}
Bolker, Benjamin M. 2008. \emph{Ecological Models and Data in R}. 1st
ed. Princeton, New Jersey: Princeton University Press.

\hypertarget{ref-Cohen1992}{}
Cohen, Jacob. 1992. ``A power primer.'' \emph{Psychological Bulletin}
112 (1): 155--59.
doi:\href{https://doi.org/10.1037/0033-2909.112.1.155}{10.1037/0033-2909.112.1.155}.

\hypertarget{ref-Field2007}{}
Field, Scott A, Patrick J O Connor, Andrew J Tyre, and Hugh P
Possingham. 2007. ``Making monitoring meaningful.'' \emph{Austral
Ecology} 32: 485--91.
doi:\href{https://doi.org/10.1111/j.1442-9993.2007.01715.x}{10.1111/j.1442-9993.2007.01715.x}.

\hypertarget{ref-Gerber1999}{}
Gerber, L R, D P DeMaster, and P M Kareiva. 1999. ``Gray whales and the
value of monitering data in implementing the U.S. endangered species
act.'' \emph{Conservation Biology} 13 (5): 1215--9.

\hypertarget{ref-Gerrodette1987}{}
Gerrodette, Tim. 1987. ``A power analysis for detecting trends.''
doi:\href{https://doi.org/10.2307/1939220}{10.2307/1939220}.

\hypertarget{ref-Giron-Nava2017}{}
Giron-Nava, Alfredo, Chase C James, Andrew F Johnson, David Dannecker,
Bethany Kolody, Adrienne Lee, Maitreyi Nagarkar, et al. 2017.
``Quantitative argument for long-term ecological monitoring.''
\emph{Marine Ecology Progress Series} 572: 269--74.

\hypertarget{ref-Hatch2003}{}
Hatch, S A. 2003. ``Statistical power for detecting trends with
applications to seabirds monitoring.'' \emph{Biological Conservation}
111: 317--29.

\hypertarget{ref-Hauser2006}{}
Hauser, Cindy E., Anthony R. Pople, and Hugh P. Possingham. 2006.
``Should managed populations be monitored every year?'' \emph{Ecological
Applications} 16 (2): 807--19.

\hypertarget{ref-Hughes2017}{}
Hughes, Brent B, Rodrigo Beas-luna, Allison K Barner, Kimberly Brewitt,
Daniel R Brumbaugh, Elizabeth B. Cerny-Chipman, Sarah L. Close, et al.
2017. ``Long-term studies contribute disproportionately to ecology and
policy.'' \emph{BioScience} 67 (3): 271--81.
doi:\href{https://doi.org/10.1093/biosci/biw185}{10.1093/biosci/biw185}.

\hypertarget{ref-Hutchings2012}{}
Hutchings, Jeffrey A, Ransom A Myers, Verónica B García, Luis O
Lucifora, and Anna Kuparinen. 2012. ``Life-history correlates of
extinction risk and recovery potential.'' \emph{Ecological Applications}
22 (4): 1061--7.

\hypertarget{ref-IUCN2012}{}
IUCN. 2012. ``IUCN Red List Categories and Criteria: Version 3.1.''
doi:\href{https://doi.org/10.9782-8317-0633-5}{10.9782-8317-0633-5}.

\hypertarget{ref-Johnson2015}{}
Johnson, Paul CD, Sarah JE Barry, Heather M Ferguson, and Pie Müller.
2015. ``Power analysis for generalized linear mixed models in ecology
and evolution.'' \emph{Methods in Ecology and Evolution} 6 (2): 133--42.
doi:\href{https://doi.org/10.1111/2041-210X.12306}{10.1111/2041-210X.12306}.

\hypertarget{ref-Keith2015}{}
Keith, David, H. Resit Akçakaya, Stuart H.M. Butchart, Ben Collen,
Nicholas K. Dulvy, Elizabeth E. Holmes, Jeffrey A. Hutchings, et al.
2015. ``Temporal correlations in population trends: Conservation
implications from time-series analysis of diverse animal taxa.''
\emph{Biological Conservation} 192. Elsevier B.V.: 247--57.
doi:\href{https://doi.org/10.1016/j.biocon.2015.09.021}{10.1016/j.biocon.2015.09.021}.

\hypertarget{ref-Legg2006}{}
Legg, Colin J, and Laszlo Nagy. 2006. ``Why most conservation monitoring
is, but need not be, a waste of time.'' \emph{Journal of Environmental
Management} 78: 194--99.
doi:\href{https://doi.org/10.1016/j.jenvman.2005.04.016}{10.1016/j.jenvman.2005.04.016}.

\hypertarget{ref-Magurran2010}{}
Magurran, Anne E, Stephen R Baillie, Stephen T Buckland, Jan Mcp Dick,
David A Elston, E Marian Scott, Rognvald I Smith, Paul J Somerfield, and
Allan D Watt. 2010. ``Long-term datasets in biodiversity research and
monitoring : assessing change in ecological communities through time.''
\emph{Trends in Ecology and Evolution} 25: 574--82.
doi:\href{https://doi.org/10.1016/j.tree.2010.06.016}{10.1016/j.tree.2010.06.016}.

\hypertarget{ref-Mapstone1995}{}
Mapstone, Bruce D. 1995. ``Scalable decision rules for environmental
impact studies : effect Size , type I , and type II errors.''
\emph{Ecological Applications} 5 (2): 401--10.

\hypertarget{ref-McDonald-Madden2010}{}
McDonald-Madden, Eve, Peter W J Baxter, Richard A. Fuller, Tara G.
Martin, Edward T. Game, Jensen Montambault, and Hugh P. Possingham.
2010. ``Monitoring does not always count.'' \emph{Trends in Ecology and
Evolution} 25 (10): 547--50.
doi:\href{https://doi.org/10.1016/j.tree.2010.07.002}{10.1016/j.tree.2010.07.002}.

\hypertarget{ref-Morrison2008}{}
Morrison, Shawm, and David S. Hik. 2008. ``When? Where? And for how
long? Census design considerations for an Alpine Lagomorph, the Collared
pika.'' In \emph{Lagomorph Biology}, 103--13. Springer Berlin
Heidelberg.
doi:\href{https://doi.org/10.1007/978-3-540-72446-9}{10.1007/978-3-540-72446-9}.

\hypertarget{ref-Myhrvold2015}{}
Myhrvold, Nathan P., Elita Baldridge, Benjamin Chan, Dhileep Sivam,
Daniel L. Freeman, and S.K. Morgan Ernest. 2015. ``An amniote
life-history database to perform comparative analyses with birds,
mammals, and reptiles.'' \emph{Ecology} 96 (11): 3109.

\hypertarget{ref-GPDD2010}{}
NERC Centre for Population Biology, Imperial College. 2010. ``The Global
Population Dynamics Database Version 2.''

\hypertarget{ref-Nichols2006}{}
Nichols, James D., and Bryon K. Williams. 2006. ``Monitoring for
conservation.'' \emph{Trends in Ecology and Evolution} 21 (12): 668--73.
doi:\href{https://doi.org/10.1016/j.tree.2006.08.007}{10.1016/j.tree.2006.08.007}.

\hypertarget{ref-RCoreTeam2016}{}
R Core Team. 2016. ``R: A language and environment for statistical
computing.'' Vienna, Austria: R Foundation for Statistical Computing.
\url{https://www.r-project.org/}.

\hypertarget{ref-Rhodes2011}{}
Rhodes, Jonathan R., and Niclas Jonzen. 2011. ``Monitoring temporal
trends in spatially structured populations: how should sampling effort
be allocated between space and time?'' \emph{Ecography} 34 (6): 1040--8.
doi:\href{https://doi.org/10.1111/j.1600-0587.2011.06370.x}{10.1111/j.1600-0587.2011.06370.x}.

\hypertarget{ref-Rueda-Cediel2015}{}
Rueda-Cediel, Pamela, Kurt E Anderson, Tracey J Regan, Janet Franklin,
and M Regan. 2015. ``Combined influences of model choice, data quality,
and data quantity when estimating population trends.'' \emph{PLoSONE} 10
(7): e0132255.
doi:\href{https://doi.org/10.1371/journal.pone.0132255}{10.1371/journal.pone.0132255}.

\hypertarget{ref-Seavy2007}{}
Seavy, Nathaniel E., and Michelle H. Reynolds. 2007. ``Is statistical
power to detect trends a good assessment of population monitoring?''
\emph{Biological Conservation} 140 (1-2): 187--91.
doi:\href{https://doi.org/10.1016/j.biocon.2007.08.007}{10.1016/j.biocon.2007.08.007}.

\hypertarget{ref-Wagner2009}{}
Wagner, Tyler, Christopher S. Vandergoot, and Jeff Tyson. 2009.
``Evaluating the power to detect temporal trends in fishery-independent
surveys - A case study based on gill nets set in the Ohio waters of Lake
Erie for walleyes.'' \emph{North American Journal of Fisheries
Management} 29: 805--16.
doi:\href{https://doi.org/10.1577/M08-197.1}{10.1577/M08-197.1}.

\hypertarget{ref-White2014}{}
White, Easton R, John D Nagy, and Samuel H Gruber. 2014. ``Modeling the
population dynamics of lemon sharks.'' \emph{Biology Direct} 9 (23):
1--18.

\hypertarget{ref-Zuur2009}{}
Zuur, Alain F., Elena N. Ieno, Neil J. Walker, Anatoly A. Saveliev, and
Graham M. Smith. 2009. \emph{Mixed Effects Models and Extensions in
Ecology with R}. New York: Springer.

\clearpage

\section{Supplementary material}\label{supplementary-material}

\setcounter{figure}{0} \renewcommand{\thefigure}{A\arabic{figure}}

\begin{figure}[htbp]
\centering
\caption{(a) Population size of Bigeye tuna (\emph{Thunnus obesus}) over
time. The line is the best fit line from linear regression. (b)
Statistical power for different subsets of the time series in panel
a.\label{fig:empirical_approach_example}}
\end{figure}

\begin{figure}[htbp]
\centering
\caption{Output of generalized linear model with a Poisson error
structure for predicting the minimum time required with explanatory
variables of the absolute value of the slope coefficient (or trend
strength), temporal autocorrelation, and variability in population
size.\label{fig:poisson_model}}
\end{figure}

\begin{figure}[htbp]
\centering
\caption{Minimum time required versus (a) generation length (years), (b)
litter size (n), (c) adult body mass (grams), (d) maximum longevity
(years), (e) egg mass (grams), and (f) incubation (days). The lines in
each plot represent the best fit line from linear
regression.\label{fig:biological_correlates}}
\end{figure}

\begin{figure}[htbp]
\centering
\caption{(a) Minimum time required to estimate change in abundance for
species class, (b) long-term trend (estimated slope coefficient) by
species class, (c) inter-annual variability in population size by species
class, and (d) temporal autocorrelation by species
class.\label{fig:class}}
\end{figure}

\begin{figure}[htbp]
\centering
\caption{Minimum time required to assess long-term trends in abundance
for values of statistical significance (\(\alpha\)) and power
(\(1-\beta\)).\label{fig:min_time_vs_alpha_beta}}
\end{figure}

\begin{figure}[htbp]
\centering
\caption{Distribution of the minimum time required in order to detect a
significant trend (at the 0.05 level) in log(abundance) given power of
0.8.\label{fig:min_time_dist_log_pop}}
\end{figure}

\begin{figure}[htbp]
\centering
\caption{Statistical power for different length of time series
simulations for a lemon shark population in Bimini,
Bahamas.\label{fig:shark_example}}
\end{figure}

\begin{figure}[htbp]
\centering
\caption{Example of calculating minimum time required for growth rate
estimation. (a) European herring gull (\emph{Larus argentatus}) scaled
density over time, (b) mean and standard deviation of growth rate for
subsamples of entire time series, and (c) the percent error between mean
estimated growth rate and the true long-term growth rate. The vertical
bar denotes the minimum time required to estimate growth rate within
20\% error.\label{fig:growth_rate}}
\end{figure}

\begin{figure}[htbp]
\centering
\caption{Histogram of the minimum time required in order to estimate the
long-term growth rate within 20\%
error.\label{fig:min_time_growth_dist}}
\end{figure}

\begin{figure}[htbp]
\centering
\caption{(a) Time series for Bigeye tuna (\emph{Thunnus obesus}) with
corresponding fitted GAM model in red and (b) statistical power as a
function of the number of years sampled. The horizontal line at 0.8
indicates the minimum threshold for statistical power and the vertical
line denotes the minimum time required to achieve 0.8 statistical
power.\label{fig:gam_example}}
\end{figure}

\begin{figure}[htbp]
\centering
\caption{Distribution of the minimum time required in order to detect a
significant trend (at the 0.05 level) in abundance according to a GAM
model given statistical power of 0.8. The smoothing parameter was set to
3 for each population.\label{fig:min_time_dist_gam}}
\end{figure}


\end{document}
